\pdfbookmark{Общая характеристика работы}{characteristic}             % Закладка pdf
\section*{Общая характеристика работы}

\newcommand{\actuality}{\pdfbookmark[1]{Актуальность}{actuality}\underline{\textbf{\actualityTXT}}}
\newcommand{\progress}{\pdfbookmark[1]{Разработанность темы}{progress}\underline{\textbf{\progressTXT}}}
\newcommand{\aim}{\pdfbookmark[1]{Цели}{aim}\underline{{\textbf\aimTXT}}}
\newcommand{\tasks}{\pdfbookmark[1]{Задачи}{tasks}\underline{\textbf{\tasksTXT}}}
\newcommand{\aimtasks}{\pdfbookmark[1]{Цели и задачи}{aimtasks}\aimtasksTXT}
\newcommand{\novelty}{\pdfbookmark[1]{Научная новизна}{novelty}\underline{\textbf{\noveltyTXT}}}
\newcommand{\influence}{\pdfbookmark[1]{Практическая значимость}{influence}\underline{\textbf{\influenceTXT}}}
\newcommand{\methods}{\pdfbookmark[1]{Методология и методы исследования}{methods}\underline{\textbf{\methodsTXT}}}
\newcommand{\defpositions}{\pdfbookmark[1]{Положения, выносимые на защиту}{defpositions}\underline{\textbf{\defpositionsTXT}}}
\newcommand{\reliability}{\pdfbookmark[1]{Достоверность}{reliability}\underline{\textbf{\reliabilityTXT}}}
\newcommand{\probation}{\pdfbookmark[1]{Апробация}{probation}\underline{\textbf{\probationTXT}}}
\newcommand{\contribution}{\pdfbookmark[1]{Личный вклад}{contribution}\underline{\textbf{\contributionTXT}}}
\newcommand{\publications}{\pdfbookmark[1]{Публикации}{publications}\underline{\textbf{\publicationsTXT}}}


{\actuality} 
В настоящее время диссипативные квантовые системы являются неотъемлемой частью экспериментальной и технологической реальности.
Практические реализации нано- и оптомеханических систем, сверхпроводящих элементов не осуществимы в полной изоляции от окружающей среды.
Подобные системы являются открытыми и их динамика существенно диссипативна, а преимущественно используемое описание посредством эволюции когерентных квантовых систем является приближением, часто грубым, реальных физических процессов.
Диссипация в данных системах является полноправным генератором эволюции, которая не менее сложна и разнообразна, чем унитарная, генерируемая гамильтонианами.
В частности, асимптотические состояния открытых квантовых систем определяются не только гамильтоновой динамикой, но взаимодействием с окружающей средой.
Вместе с тем, следует отметить, что данное взаимодействие не является достаточно сильным, чтобы полностью свести динамику системы к классическому случаю \autocite{Breuer2007}.
Более того, особенностью диссипации является то, что она способна привести систему в состояния, которые недостижимы в классическом пределе.
Здесь, в первую очередь, следует привести такие примеры, как новые топологические состояния, получаемые за счёт управляемой синтетической диссипации \autocite{Diehl2011} или «чистые» сильно запутанные состояния в многочастичных квантовых системах \autocite{Kraus2008}.

На сегодняшний день изучение открытых квантовых систем является активно развивающейся областью теоретической и экспериментальной физики с широким спектром разноплановых фундаментальных результатов и методов.
Эксперименты с открытыми квантовыми системами являются трудоёмкими и могут быть выполнены только в специальных, весьма ограниченных условиях.
Поэтому основным подходом теоретического изучения физики открытых квантовых систем является математическое моделирование.
В частности, одним из самых распространённых подходов к описанию динамики открытых квантовых систем является формализм Линдблада \autocite{Lindblad1976}.
Поиск аналитических решений существующих математических моделей осложнён чрезвычайно высокой размерностью получаемых систем дифференциальных уравнений.
Из-за этого основным инструментом изучения открытых квантовых систем является численное моделирование.
Актуальной также является разработка новых численных характеристик и методов анализа.
Среди активно развиваемых и наиболее удачных подходов с точки зрения параллелизма вычислений следует выделить метод квантовых скачков \autocite{Dalibard1992}.
Данный подход является популярным инструментом в сфере квантовой оптики \autocite{Carmichael1993}, имеет приложения в вычислительной физике холодных атомов \autocite{Diehl2011} и активно используется в контексте квантовых электродинамических (КЭД) систем \autocite{Arakawa2015}.

В теории открытых квантовых систем существует ряд фундаментальных проблем, которые ещё не получили должного математического описания.
В центре внимания данной диссертационной работы находятся две из них: локализация и хаос.
Данные проблемы являются актуальными и активно обсуждаются на различных конференциях мирового уровня, в частности «Dissipative Quantum Chaos: from Semi-Groups to QED Experiments» (Daejeon, South Korea, октябрь 2017) и «Quantization of Dissipative Chaos: Ideas and Means» (Bad Honnef, Germany, декабрь 2019).

Теория локализации (одночастичной и многочастичной) является хорошо развитой для изолированных квантовых систем, а также нелинейных классических систем, получаемых из квантовых в приближении среднего поля.
Есть указания на то, что свойства локализации могут проявляться в открытых квантовых системах, но все, что было обнаружено в предшествующих исследованиях, касается переходных процессов\autocite{Genway2014}.
Возникает целый ряд вопросов о проявлениях одночастичной и многочастичной локализации в асимптотических состояниях открытых квантовых систем, о специфике математических моделей, в которых это может проявляться, о численных методах, позволяющих это обнаружить. 
Ответы на данные вопросы до недавнего времени оставались неизвестными.

Теория диссипативного квантового хаоса переживает активное развитие и напрямую связана с распространением методов и подходов классической теории бифуркаций и детерминированного хаоса.
В частности, в работе \autocite{Ivanchenko2017} было обобщено понятие бифуркации для случая квантовых диссипативных систем и обнаружены некоторые аналоги классических бифуркаций.
Однако, возникают вопросы о существовании других классических бифуркационных сценариев в открытых квантовых системах, о математических моделях в которых они могут быть обнаружены, а также о методах исследования квантовых бифуркаций.
Важнейшим направлением исследований является создание математических методов (и их программная реализация) для количественной оценки диссипативного квантового хаоса.
Отдельный интерес представляет исследование математических моделей, описывающих физические системы, в которых количественные характеристики квантового хаоса можно получить из данных реального эксперимента.

Из всего вышесказанного вытекает актуальность исследования локализации и хаоса в диссипативных квантовых системах. 
Помимо теоретического значения, эти результаты важны для реализации новых рабочих режимов и управления этими режимами в таких квантовых устройствах, как квантовые электродинамические (КЭД) резонаторы \autocite{Arakawa2015} и сверхпроводящие цепи, которые являются существенно диссипативными.

{\aim} данной работы является развитие математических моделей, численных методов и их программных реализаций, позволяющих получить, исследовать и охарактеризовать явления локализации и хаоса в открытых квантовых системах.

Для~достижения поставленной цели необходимо было решить следующие {\tasks}:
\begin{enumerate}[beginpenalty=10000] % https://tex.stackexchange.com/a/476052/104425
	\item Разработать программный комплекс для численного моделирования динамики открытых квантовых систем с большим числом состояний, включающий в себя возможность анализа отдельных квантовых траекторий, поиск асимптотических состояний системы путём численного интегрирования (при наличии модуляции в системе) или поиска собственного состояния системы, если модуляции нет.
	\item Идентифицировать и численно исследовать математические модели открытых квантовых систем, которые допускают существование одночастичной локализации в асимптотических состояниях. Разработать методы управления локализационными свойствами таких систем. Численно моделировать распространения волновых пакетов в системах с одночастичной локализацией, выявить особенности данных процессов в зависимости от степени локализации.
	\item Идентифицировать и численно моделировать открытые квантовые системы с многочастичной локализацией в асимптотических состояниях, исследовать количественные признаки данного явления.
	\item Идентифицировать и численно исследовать математические модели открытых квантовых систем, в которых существуют новые квантовые аналоги классических бифуркаций.
	\item Разработать численный алгоритм поиска старшего квантового показателя Ляпунова, позволяющего количественно оценить динамику открытой квантовой системы.
	\item Разработать квантификаторы диссипативного квантового хаоса, которые могут быть доступны в реальном эксперименте. Численно исследовать математические модели соответствующих открытых квантовых систем. Изучить способы управления динамикой открытых квантовых систем на основе спин-фотонного взаимодействия. 
\end{enumerate}


{\novelty}
\begin{enumerate}[beginpenalty=10000] % https://tex.stackexchange.com/a/476052/104425
	\item Впервые предложена и численно исследована математическая модель открытой квантовой системы с признаками одночастичной локализации в асимптотических состояниях \cite{Yusipov2017}. 
	Предложен метод управления структурой квантового аттрактора с признаками локализации, используя фазовые свойства экспериментально реализуемой и управляемой диссипации. 
	Установлена устойчивость локализации к дефазирующей диссипации \cite{Vershinina2017}.
	Численно исследована зависимость типа распространения волновых пакетов в открытых квантовых системах с локализацией от типа управляемой диссипации \cite{Yusipov2018}.
	\item Впервые предложена и исследована математическая модель открытой квантовой системы с признаками многочастичной локализации. Предложены численные характеристики данного явления \cite{Vakulchyk2018}.
	\item Численно исследована модель открытого квантового димера, в которой впервые был обнаружен квантовый аналог классической бифуркации Неймарка"--~Сакера \cite{Yusipov2019_1} (рождение инвариантного тора из-за неустойчивости предельного цикла).
	\item Впервые предложен и реализован алгоритм вычисления старшего квантового показателя Ляпунова на основе метода квантовых траекторий \cite{Yusipov2019_2}.
	\item Предложены новые количественные характеристики диссипативного квантового хаоса, которые могут наблюдаться в реальном физическом эксперименте \cite{Yusipov2020}. Численно исследована модель открытого квантового резонатора со спином, в которой существует возможность контролировать тип динамики системы \cite{Yusipov2021}.
\end{enumerate}

{\influence}. Благодаря очень быстрому прогрессу в области прикладных квантовых вычислений (все существующие на данный момент квантовые чипы, выпущенные компаниями Google, Intel, IBM, и D-Wave, состоят из сверхпроводящих кубитов), сверхпроводящие структуры оказались на переднем крае квантовой технологии. 
Сверхпроводящие чипы используются теперь не только для выполнения квантовых вычислений, но и для изучения многочастичной локализации \autocite{Roushan2017} и разнообразных фаз квантовых многочастичных систем, таким образом предоставляя новый, хорошо контролируемый и настраиваемый стенд для изучения сложных квантовых систем \autocite{Barends2015}. 
Исследование диссипативного квантового хаоса и его механизмов позволит использовать диссипативные эффекты (вместо того, чтобы бороться и подавлять их) для создания принципиально новых режимов сверхпроводящих квантовых систем, решающих задачу устойчивой обработки квантовой информации на длительных временных масштабах.

{\methods} 
Математические модели открытых квантовых систем были получены при в рамках формализма Линдблада \autocite{book2007} для матрицы плотности.
Численный анализ полученных моделей осуществлялся посредством прямого численного интегрирования (методы высоких порядков) уравнения Линдблада в случае, если в системе есть модуляция или посредством численного поиска собственного вектора матрицы оператора Линдблада, соответствующего нулевому собственному числу  (если модуляции нет). При этом для вычислений использовался как классический базис гильбертова пространства состояний квантовой системы, так и специальный базис \cite{Liniov2019}, состоящий из обобщения матриц Гелл-Манна на любое количество состояний (генераторы SU(N) групп).
Также применялось микроскопическое описание открытых квантовых систем в терминах отдельных квантовых траекторий, что позволяло  «развернуть» динамику на асимптотическом квантовом состоянии "--- аттракторе "--- в ансамбль независимых реализаций (траекторий) \autocite{Plenio1998}.

{\defpositions}
\begin{enumerate}[beginpenalty=10000] % https://tex.stackexchange.com/a/476052/104425
	\item 
	Реализован программный комплекс на языке C++, предназначенный для численного исследования открытых квантовых систем. Асимптотические состояния находятся путём численного интегрирования уравнения Линдблада или путём решения задачи по поиску собственных векторов и значений для матрицы оператора Линдблада. Программный комплекс предусматривает возможность моделирования и анализа отдельных квантовых траекторий.
	\item 
	Предложена математическая модель открытой квантовой системы с признаками одночастичной локализации в асимптотических состояниях \cite{Yusipov2017}. Свойства локализации квантового аттрактора определяются фазовым параметром диссипативных операторов \cite{Vershinina2017}. При этом распространение волновых пакетов отдельных квантовых траекторий в асимптотическом режиме может иметь как диффузионный, так и баллистический характер \cite{Yusipov2018}.
	\item 
	Предложена математическая модель открытой квантовой системы с признаками  многочастичной локализации. Переход между локализацией и делокализацией в численном эксперименте сопровождается изменениями в параметрах дисбаланса, в энтропии запутанности операторного пространства и в статистике собственных значений асимптотический матрицы плотности \cite{Vakulchyk2018}.
	\item 
	В математической модели открытого периодически модулируемого квантового димера наблюдается квантовый аналог классической бифуркации Неймарка"--~Сакера (рождение тора из-за неустойчивости предельного цикла). Численное моделирование квантовых траекторий позволяет исследовать динамику на квантовом торе и, в частности, определить число вращения \cite{Yusipov2019_1}.
	\item 
	Разработан и программно реализован \cite{prog1} новый численный алгоритм нахождения старшего квантового показателя Ляпунова, основанный на методе квантовых траекторий. Данный подход позволяет численно исследовать структуру регулярных и хаотических областей в пространстве параметров открытой квантовой системы. Как и классический старший показатель Ляпунова, его квантовый аналог становится положительным в случае хаотической динамики системы \cite{Yusipov2019_2}.
	\item 
	По крайней мере в некоторых математических моделях открытых квантовых систем в регулярном и хаотическом режимах наблюдается качественно различная статистика распределения времени между излучением системой отдельных фотонов, которая является экспериментально измеримой характеристикой. А именно, при переходе в режим квантового хаоса распределение времени ожидания фотона становится существенно не пуассоновским, появляется степенная асимптотика \cite{Yusipov2020}.
	\item
	Изменение силы спин-фотонного взаимодействия в системе открытого оптического резонатора с периодической модуляцией может обусловливать переходы между регулярными и хаотическими режимами\cite{Yusipov2021}.
\end{enumerate}

{\reliability} полученных результатов обеспечивается применением современных и принятых в научном сообществе методов численного моделирования физики открытых квантовых систем, а также сравнением результатов с работами других авторов. 
Результаты численных экспериментов полностью согласуются с теорией.

{\probation}
Основные результаты работы докладывались~на:

\begin{itemize}
	\item 22-ая Нижегородская сессия молодых учёных (естественные, математические науки) (Россия, Нижний Новгород, 23--26 мая 2017) \cite{sessiann_2017};
	\item XXI научная конференции по радиофизике (Россия, Нижний Новгород, 15--22 мая 2017) \cite{rf_2017};
	\item Третий Всероссийский молодёжный научный форум «Наука будущего – наука молодых» (Россия, Нижний Новгород, 12--15 сентября 2017) \cite{sfy_2017};
	\item Международная конференция «Shilnikov WorkShop 2017» (Россия, Нижний Новгород, 15--16 декабря 2017) \cite{shilnikov_2017};
	\item 23-я Нижегородская сессия молодых учёных (технические, естественные, математические науки) (Россия, Нижний Новгород, 22--23 мая 2017) \cite{sessiann_2018};
	\item XXII научная конференция по радиофизике, посвященная 100-летию Нижегородской радиолаборатории (Россия, Нижний Новгород, 15--29 мая 2018) \cite{rf_2018};
	\item XIII Всероссийская конференции молодых учёных «Наноэлектроника, нанофотоника и нелинейная физика» (Россия, Саратов, 4--6 сентября 2018) \cite{nnnph_2018};
	\item 9th International Scientific Conference on Physics and Control (PhysCon2019) (Россия, Иннополис, 8--11 сентября 2019) \cite{physcon_2019};
	\item International Conference «Quantization of Dissipative Chaos: Ideas and Means» (Germany, Bad-Honnef, 16--20 декабря 2019).
	\item XXIV научная конференция по радиофизике, посвященная 75-летию радиофизического факультета (Россия, Нижний Новгород, 13--31 мая 2020) \cite{rf_2020};
\end{itemize}

{\contribution} Все представленные в работе результаты были либо
получены лично автором, либо при его непосредственном участии. Автор принимал прямое участие в постановке задач, получении и анализе полученных результатов, а также в подготовке публикаций в научных журналах и представлении докладов на тематических конференциях.

\ifnumequal{\value{bibliosel}}{0}
{%%% Встроенная реализация с загрузкой файла через движок bibtex8. (При желании, внутри можно использовать обычные ссылки, наподобие `\cite{vakbib1,vakbib2}`).
    {\publications} Основные результаты по теме диссертации изложены
    в~XX~печатных изданиях,
    X из которых изданы в журналах, рекомендованных ВАК,
    X "--- в тезисах докладов.
}%
{%%% Реализация пакетом biblatex через движок biber
    \begin{refsection}[bl-author, bl-registered]
        % Это refsection=1.
        % Процитированные здесь работы:
        %  * подсчитываются, для автоматического составления фразы "Основные результаты ..."
        %  * попадают в авторскую библиографию, при usefootcite==0 и стиле `\insertbiblioauthor` или `\insertbiblioauthorgrouped`
        %  * нумеруются там в зависимости от порядка команд `\printbibliography` в этом разделе.
        %  * при использовании `\insertbiblioauthorgrouped`, порядок команд `\printbibliography` в нём должен быть тем же (см. biblio/biblatex.tex)
        %
        % Невидимый библиографический список для подсчёта количества публикаций:
        \printbibliography[heading=nobibheading, section=1, env=countauthorvak,          keyword=biblioauthorvak]%
        \printbibliography[heading=nobibheading, section=1, env=countauthorwos,          keyword=biblioauthorwos]%
        \printbibliography[heading=nobibheading, section=1, env=countauthorscopus,       keyword=biblioauthorscopus]%
        \printbibliography[heading=nobibheading, section=1, env=countauthorconf,         keyword=biblioauthorconf]%
        \printbibliography[heading=nobibheading, section=1, env=countauthorother,        keyword=biblioauthorother]%
        \printbibliography[heading=nobibheading, section=1, env=countregistered,         keyword=biblioregistered]%
        \printbibliography[heading=nobibheading, section=1, env=countauthorpatent,       keyword=biblioauthorpatent]%
        \printbibliography[heading=nobibheading, section=1, env=countauthorprogram,      keyword=biblioauthorprogram]%
        \printbibliography[heading=nobibheading, section=1, env=countauthor,             keyword=biblioauthor]%
        \printbibliography[heading=nobibheading, section=1, env=countauthorvakscopuswos, filter=vakscopuswos]%
        \printbibliography[heading=nobibheading, section=1, env=countauthorscopuswos,    filter=scopuswos]%
        %
        \nocite{*}%
        %
        {\publications} Основные результаты по теме диссертации изложены в~\arabic{citeauthor}~печатных изданиях\sloppy%
        \ifnum \value{citeauthorscopuswos}>0%
        	, \arabic{citeauthorscopuswos} "--- в~периодических научных журналах, индексируемых Web of~Science и Scopus\sloppy%
        \fi%
        \ifnum \value{citeauthorconf}>0%
            , \arabic{citeauthorconf} "--- в~тезисах докладов.
        \else%
            .
        \fi%
        \ifnum \value{citeregistered}=1%
            \ifnum \value{citeauthorpatent}=1%
                Зарегистрирован \arabic{citeauthorpatent} патент.
            \fi%
            \ifnum \value{citeauthorprogram}=1%
                Зарегистрирована \arabic{citeauthorprogram} программа для ЭВМ.
            \fi%
        \fi%
        \ifnum \value{citeregistered}>1%
            Зарегистрированы\ %
            \ifnum \value{citeauthorpatent}>0%
            \formbytotal{citeauthorpatent}{патент}{}{а}{}\sloppy%
            \ifnum \value{citeauthorprogram}=0 . \else \ и~\fi%
            \fi%
            \ifnum \value{citeauthorprogram}>0%
            \formbytotal{citeauthorprogram}{программ}{а}{ы}{} для ЭВМ.
            \fi%
        \fi%
        % К публикациям, в которых излагаются основные научные результаты диссертации на соискание учёной
        % степени, в рецензируемых изданиях приравниваются патенты на изобретения, патенты (свидетельства) на
        % полезную модель, патенты на промышленный образец, патенты на селекционные достижения, свидетельства
        % на программу для электронных вычислительных машин, базу данных, топологию интегральных микросхем,
        % зарегистрированные в установленном порядке.(в ред. Постановления Правительства РФ от 21.04.2016 N 335)
    \end{refsection}%
    \begin{refsection}[bl-author, bl-registered]
        % Это refsection=2.
        % Процитированные здесь работы:
        %  * попадают в авторскую библиографию, при usefootcite==0 и стиле `\insertbiblioauthorimportant`.
        %  * ни на что не влияют в противном случае
        %\nocite{vakbib2}%vak
        %\nocite{patbib1}%patent
        %\nocite{progbib1}%program
        %\nocite{bib1}%other
        %\nocite{confbib1}%conf
    \end{refsection}%
        %
        % Всё, что вне этих двух refsection, это refsection=0,
        %  * для диссертации - это нормальные ссылки, попадающие в обычную библиографию
        %  * для автореферата:
        %     * при usefootcite==0, ссылка корректно сработает только для источника из `external.bib`. Для своих работ --- напечатает "[0]" (и даже Warning не вылезет).
        %     * при usefootcite==1, ссылка сработает нормально. В авторской библиографии будут только процитированные в refsection=0 работы.
} % Характеристика работы по структуре во введении и в автореферате не отличается (ГОСТ Р 7.0.11, пункты 5.3.1 и 9.2.1), потому её загружаем из одного и того же внешнего файла, предварительно задав форму выделения некоторым параметрам

%Диссертационная работа была выполнена при поддержке грантов \dots

%\underline{\textbf{Объем и структура работы.}} Диссертация состоит из~введения,
%четырех глав, заключения и~приложения. Полный объем диссертации
%\textbf{ХХХ}~страниц текста с~\textbf{ХХ}~рисунками и~5~таблицами. Список
%литературы содержит \textbf{ХХX}~наименование.

\pdfbookmark{Содержание работы}{description}                          % Закладка pdf
\section*{Содержание работы}
Во \underline{\textbf{введении}} обосновывается актуальность исследований, проводимых в~рамках данной диссертационной работы, формулируется цель, ставятся задачи работы, излагается научная новизна
и практическая значимость представляемой работы, приводятся положения, выносимые на защиту. 
Введение содержит сведения о достоверности и апробации результатов.



\underline{\textbf{Первая глава}} посвящена описанию математических моделей открытых квантовых систем.
Сначала приводятся основные элементы описания квантовых систем: пространство состояний, операции, уравнение Шрёдингера и фон Неймана, операторы плотности, особенности квантовых измерений.
После этого рассматривается основной предмет исследования "--- открытые квантовые системы, которые описываются уравнением Линдблада для матрицы плотности:
\begin{equation}
	\label{eq:GKSL_base}
	\begin{gathered}
		\frac{\partial}{\partial t} \rho (t) = \mathcal{L}(\rho(t), t) = -i \left[ H(t), \rho(t) \right] + \sum_{k=1}^{K} \gamma_{k}(t) \mathcal{D}_k(t), \\
		\mathcal{D}_k(t) =  V_k(t) \rho(t) V_k^\dagger(t) - \frac{1}{2} \left\lbrace V_k^\dagger(t) V_k(t), \rho(t) \right\rbrace ,
	\end{gathered}
\end{equation}
где первое слагаемое в правой части первого уравнения является унитарной частью, которая отвечает за когерентную эволюцию системы с гамильтонианом \(H(t)\), а второе слагаемое "--- диссипативная часть, отвечающая за взаимодействие с окружающей средой. 
Диссипация осуществляется через \(K\) каналов, каждый из которых характеризуется скоростью диссипации \(\gamma_{k}(t)\) и непосредственно диссипативным оператором (диссипатором) \(V_k(t)\).  
Для моделирования конкретной открытой квантовой системы нужно указать вид операторов для обеих частей уравнения Линдблада (унитарной и диссипативной).

Основная вычислительная задача, решаемая при исследовании открытых квантовых систем "--- отыскание асимптотической матрицы плотности, эволюция которой определяется уравнением \cref{eq:GKSL_base}. 
Есть три общепринятых пути ее вычисления:
\begin{enumerate}[beginpenalty=10000] % https://tex.stackexchange.com/a/476052/104425
	\item Спектральные методы "--- полная или частичная диагонализация линдбладиана различными видами итерационных алгоритмов (если в системе отсутствует модуляция);
	\item Численное интегрирование уравнения \cref{eq:GKSL_base} схемами высоких порядков (при наличии модуляции);
	\item Метод квантовых траекторий, позволяющий свести задачу численного решения уравнения \cref{eq:GKSL_base} к задаче статистического семплирования отдельных квантовых траекторий, уравнения для которых содержат на порядок меньшее количество состояний.
\end{enumerate}

В главе подробно описан метод квантовых траекторий, эволюция волновых функции которых описывается уравнением:
\begin{equation}
	\label{eq:qj_schrodinger}
	\begin{gathered}
		i \frac{\partial}{\partial t} \psi_j(t) = \tilde{H}(t) \psi_j(t), \\
		\tilde{H}(t) = H(t) - \frac{i}{2} \sum_{k=1}^{K} V_k^\dagger(t) V_k(t),
	\end{gathered}
\end{equation}
где \(V_k(t)\) "--- диссипативные операторы, индуцирующие квантовые скачки.

Если в квантовой системе $S$ состояний, то размерность системы дифференциальных уравнений для соответствующей матрицы плотности (уравнение \cref{eq:GKSL_base}) равна $S^2$, в то время как размерность системы для волновой функции отдельно взятой квантовой траектории (уравнение \cref{eq:qj_schrodinger}) равна $S$. 
Это даёт значительные преимущества методу квантовых траекторий с точки зрения параллельности вычислений, так как траектории семплируются независимо. 
Метод также интересен с точки зрения численного анализа отдельных квантовых траекторий, эволюция которых может характеризовать динамику системы в целом. 

В работе приведён детальный псевдокод метода квантовых траекторий и всех необходимых функций. Также рассмотрена оптимальная с точки зрения скорости вычислений версия с использованием экспоненциальных пропагаторов в случае, когда в системе отсутствует модуляция, либо она носит кусочно-постоянный характер.

\underline{\textbf{Вторая глава}} посвящена изучению свойств одночастичной и многочастичной локализации в асимптотических состояниях открытых квантовых систем, определению специфики математических моделей, в которых эти свойства могут проявляться.
Также рассматриваются численные методы и характеристики, позволяющие охарактеризовать локализационные свойства системы. 

Основополагающая модель, в которой впервые были обнаружены следы одночастичной локализации Андерсона \cite{Yusipov2017} описывается уравнением Линдблада \cref{eq:GKSL_base} со следующим гамильтонианом:

\begin{equation}
	\label{eq:anderson_H}
	\begin{gathered}
		H = \sum_{n} \varepsilon_n b^\dagger_n b_n - \left(b^\dagger_n b_{n+1} + b^\dagger_{n+1} b_{n}\right),
	\end{gathered}
\end{equation}
и диссипативными операторами:
\begin{equation}
	\label{eq:anderson_diss_local}
	\begin{gathered}
		V_k = \left( b^\dagger_k + e^{i \alpha} b^\dagger_{k+l}\right) \left( b_k - e^{-i \alpha} b_{k+l} \right),
	\end{gathered}
\end{equation}
где \(\varepsilon_n \in \left[-\frac{W}{2}, \frac{W}{2}\right]\) "--- случайные некореллированые значения энергий на сайте \(n\), \(W\) "--- сила пространственного беспорядка, \(b_n\) и \(b^\dagger_n\) "--- операторы рождения и уничтожения бозона на сайте \(n\).
Модель описывает динамику бозона на решётке с $N$ сайтами. 
Число состояний в квантовой системе совпадает с размерностью решётки $S=N$.
Диссипативные операторы $V_k$ \cite{Diehl2008} параметризованы фазой $\alpha$ и индексом соседнего сайта $l$, на который действуют.
В случае, когда \(\alpha = 0\), диссипативный оператор синхронизирует динамику на \(k\)-ом и \((k+l)\)-ом сайте, за счёт рециркуляции антисимметричных противофазных состояний в симметричные и синфазные. 
Данный тип диссипаторов имеет экспериментальную реализацию на основе квантовых резонаторов, соединённых сверхпроводящими кубитами (фаза варьируется относительным положением кубита).

\begin{figure}[ht]
	\centerfloat{
		\hfill
		\subcaptionbox[List-of-Figures entry]{\label{fig:anderson_rho_loc-1}}{%
			\includegraphics[width=0.5\linewidth]{anderson_rho_loc_1}}
		\hfill
		\subcaptionbox{\label{fig:anderson_rho_loc-2}}{%
			\includegraphics[width=0.5\linewidth]{anderson_rho_loc_2}}
		\hfill
	}
	\legend{}
	\caption[Асимптотическая матрица плотности с локализацией Андерсона]
	{
		Абсолютные значения асимптотической матрицы плотности \(\rho^A\) в базисе гильбертова пространства (a) и в базисе собственных состояний модели Андерсона (б) для единичной реализации беспорядка. Параметры диссипации: \(\alpha=0\) и \(l=1\). Сила пространственного беспорядка \(W=1\).
	}
	\label{fig:anderson_rho_loc}
\end{figure}

В рассматриваемой модели впервые были обнаружены асимптотические состояния со следами локализации \cite{Yusipov2017}" --- матрица плотности имеет пятнистую структуру с несколькими яркими областями локализации (рисунок \cref{fig:anderson_rho_loc-1}). 
В ходе дальнейшего исследования выяснилось, что в решении доминируют несколько локализованных мод пространственно неоднородного гамильтониана из классической модели Андерсона "--- в базисе собственных состояний гамильтониана \cref{eq:anderson_H} матрица плотности является диагональной с преобладанием собственных значений из нижней границы спектра(рисунок \cref{fig:anderson_rho_loc-2}). 
В работе было проведено дополнительное аналитическое исследование диагональных элементов матрицы плотности и установлено, что формирующие решение андерсоновские моды выбираются в соответствии с их пространственно-фазовыми свойствами, унаследованными от собственных состояний гамильтониана в пределе нулевого беспорядка, в зависимости от фазы диссипативных операторов.

Также в работе было изучено влияние фазы диссипативных операторов на локализационные свойства системы и предложен механизм управления последними \cite{Vershinina2017}. 
Метод основан на фазовых свойствах локализованных мод гамильтониана системы
Андерсона, которые являются «тёмными» состояниями синтетических диссипаторов.

Кроме этого было обнаружено, что одночастичная локализация является устойчивой к дефазирующей диссипации, которая обычно приводит системы в термализованные состояния с максимальной энтропией.

\begin{figure}[h!]
	\centerfloat{
		\hfill
		\subcaptionbox[List-of-Figures entry]{\label{fig:anderson_prb_2_1}}{%
			\includegraphics[width=0.5\linewidth]{anderson_prb_2_1}}
		\hfill
		\subcaptionbox{\label{fig:anderson_prb_2_2}}{%
			\includegraphics[width=0.5\linewidth]{anderson_prb_2_2}}
		\vfill
		\subcaptionbox[List-of-Figures entry]{\label{fig:anderson_prb_2_3}}{%
			\includegraphics[width=0.5\linewidth]{anderson_prb_2_3}}
		\hfill
		\subcaptionbox{\label{fig:anderson_prb_2_4}}{%
			\includegraphics[width=0.5\linewidth]{anderson_prb_2_4}}
	}
	\legend{}
	\caption[Динамика квантовых траекторий на квантовых аттракторах в зависимости от параметров неэрмитовой диссипации]
	{
		Диагональные элементы асимптотической матрицы плотности (на левых вставках) и эволюция на аттракторе квадрата волновой функции отдельно взятой квантовой траектории (основные части на рисунках). (a): \(\alpha = 0\), в исходном базисе; (б): \(\alpha = 0\), в базисе Андерсоновских мод, отсортированных по центру масс; (в): \(\alpha=\frac{\pi}{4}\), в исходном базисе; (г): дефазирующая диссипация. 
	}
	\label{fig:anderson_prb_2}
\end{figure}
В данной главе при помощи метода квантовых траекторий была численно промоделирована и изучена микроскопическая динамика на аттракторах с различной степенью локализации (рисунок \cref{fig:anderson_prb_2}).
В случае нулевой фазы диссипации \cref{eq:anderson_diss_local} наблюдается прерывистая динамика состоящая из длительных циркуляций возле центров локализации и быстрых переходов между ними (рисунок \cref{fig:anderson_prb_2_1}). 
Если данную волновую функцию перевести в базис собственных состояний модели Андерсона, то можно заметить, что циркуляции происходят на модах Андерсона, которые формируют асимптотическое состояние равновесия (рисунок \cref{fig:anderson_prb_2_2}). 
Если рассмотреть ненулевую фазу неэрмитовых диссипаторов, то динамика резко изменится: для \(\alpha=\frac{\pi}{4}\) незначительные циркуляции возле центов локализации сохраняются, но самый существенный вклад в динамику вносит баллистическое распространение волновых пакетов (рисунок \cref{fig:anderson_prb_2_3}). 
При этом Дефазирующая диссипация не несёт какой-либо пространственно-временной структуры (рисунок \cref{fig:anderson_prb_2_4}).
В итоге было обнаружено, что в открытой квантовой системе с одночастичной локализацией существуют нетривиальные режимы распространения волновых пакетов за счёт взаимодействия беспорядка и диссипации. 
Квантовые траектории демонстрируют диффузионные (при которых циркуляция в Андерсоновских модах прерывается квантовыми скачками) и баллистические режимы. Управляя фазой неэрмитовых диссипаторов, можно переключать данные режимы. 
Также была изучена статистика времён между квантовыми скачками, которая сильно отличается в зависимости от типа распространения волновых пакетов в системе \cite{Yusipov2018}.

\begin{figure}[h]
	\centerfloat{
		\includegraphics[scale=0.3]{mbl_ratio}
	}
	\caption[Соотношение последовательных уровней для асимптотической матрицы плотности в зависимости от силы беспорядка в системе]{
		Соотношение последовательных уровней собственных значений асимптотической матрицы плотности \(\rho^A\) в зависимости от силы беспорядка \(h\). Сплошными линиями обозначено усреднённое значение \(r\) по \(N_r=100\) реализациям беспорядка для каждого значения \(h\). Области соответствующего цвета обозначают стандартное отклонение распределения \(r\). Пунктирные линии соответствуют значениям \(r\), характерным для случайных величин с распределением Пуассона (регулярная динамика, локализация) и случайных матриц из гауссова унитарного ансамбля GUE (хаос, отсутствие локализации).
	}
	\label{fig:mbl_ratio}
\end{figure}

Помимо одночастичной локализации в данной главе было численно промоделировано и изучено явление многочастичной локализации (MBL) в открытых квантовых системах. Соответствующая математическая модель задаётся уравнением Линдблада \cref{eq:GKSL_base} с многочастичным гамильтонианом для \(\frac{N}{2}\) бесспиновых фермионов на решётке размером \(N\): 
\begin{equation}
\label{eq:mbl_H}
\begin{gathered}
H = \sum_{n=1}^{N} h_n b^{\dagger}_n b_n + \sum_{n=1}^{N-1} b^{\dagger}_n b_n b^{\dagger}_{n+1} b_{n+1} - \sum_{n=1}^{N-1} \left( b^{\dagger}_n b_{n+1} + b^{\dagger}_{n+1} b_n \right) ,
\end{gathered}
\end{equation}
где \(b_n\) и \(b^{\dagger}_n\) "--- операторы уничтожения и создания фермиона на \(n\)-ом сайте решётки соответственно, \(b^{\dagger}_n b_n\) "--- оператор числа частиц на сайте \(n\).
В каждом узле решётки на фермионы действуют случайные потенциалы \(h_n\) (первое слагаемое в правой части). Фермионы, которые находятся в соседних сайтах решётки, взаимодействуют между собой (второе слагаемое в правой части). Третье слагаемое в правой части отвечает за перемещение фермионов между сайтами решётки. Значения случайных потенциалов \(h_n\) равномерно распределены в интервале \(\left[-h, h \right]\).
Как и в одночастичной модели, рассматривается специальный тип диссипации:
\begin{equation}
\label{eq:mbl_diss_diehl}
\begin{gathered}
V^l_k = ( b^\dagger_k + b^\dagger_{k+1}) \left( b_k - b_{k+1} \right),
\end{gathered}
\end{equation}
который приводит систему в состояние со следами многочастичной локализации. 
Для численной оценки многочастичной локализации в открытых квантовых системах были предложены три численных характеристики:
\begin{enumerate}[beginpenalty=10000] % https://tex.stackexchange.com/a/476052/104425
	\item статистика дисбаланса (величина, которая может быть измерена в реальном физическом эксперименте);
	\item энтропия запутанности операторного пространства (указывает на различия между эргодической фазой и многочастичной локализацией как в асимптотическом пределе, так и во время релаксации к нему);
	\item соотношение последовательных уровней \(r\) собственных значений асимптотической матрицы плотности (рисунок \ref{fig:mbl_ratio}), которое связывает явление многочастичной локализации с теорией случайных матриц и квантового хаоса.
\end{enumerate} 

В итоге было обнаружено, что физически реализуемая неэрмитовая диссипация \cref{eq:mbl_diss_diehl}, нетривиально действующая на соседних сайтах многочастичной системы пытается построить классические и квантовые корреляции между далеко расположенными сайтами, и даже дефазирующая диссипация не может их разрушить \cite{Vakulchyk2018}. 
В то же время механизмы MBL, индуцированные гамильтонианом, пытаются ограничить корреляции длиной локализации. 
В результате баланса между этими факторами возникает асимптотическое состояние со следами локализации, которые можно численно детектировать предложенными в работе характеристиками.  

Результаты данной главы опубликованы в работах \cite{Yusipov2017, Vershinina2017, Yusipov2018, Vakulchyk2018, sessiann_2017, rf_2017, shilnikov_2017}.


\underline{\textbf{Третья глава}} посвящена разработке новых численных критериев квантового диссипативного хаоса, численному исследованию математических моделей, описывающих реальные физические установки, в которых предложенные в данной главе квантификаторы хаоса могут быть измерены экспериментально. 
Также в данной главе исследуются математические модели, в которых были обнаружены новые квантовые аналоги классических бифуркационных сценариев.

Бифуркационный анализ является одним из основных подходов изучения нелинейной динамики и ее приложений.
Применение бифуркационного анализа в сфере квантовой физики долгое время рассматривалось только в контексте изолированных систем  "---  гамильтонов хаос, спектральные характеристики которого в квантовых системах к настоящему времени являются очень хорошо изученными.
Дальнейшее применение бифуркационного анализа к открытым квантовым системам в связке с соответствующими классическими нелинейными среднеполевыми моделями будет способствовать развитию теории диссипативного квантового хаоса. 
В случае открытых систем бифуркации могут иметь более серьёзные последствия, чем в гамильтоновом случае, потому что они будут влиять на стационарное состояние системы, а не только на конкретное собственное состояние изолированной системы.
Однако, существуют значительные трудности в построении среднеполевых приближений для открытых квантовых систем.
В существующих работах по данной тематике уже были обнаружены квантовые аналоги таких классических бифуркаций, как седлоузловая, типа «вилки», а также сценарий перехода к квантовому хаосу через каскад бифуркаций удвоения периода.

В данной главе численно исследовалась математическая модель открытого квантового периодически модулируемого димера, в которой впервые был обнаружен квантовый аналог классической бифуркации Неймарка"--~Сакера (рождение тора из-за неустойчивости предельного цикла) \cite{Yusipov2019_1}. Модель димера, в котором находятся $N$ бозонов, описывается уравнением Линдблада \cref{eq:GKSL_base} c Гамильтонианом следующего вида:
\begin{equation}
	\label{eq:dimer_H}
	\begin{gathered}
		H(t) = -\left(b^\dagger_1 b_2 + b^\dagger_2 b_1\right) + \frac{2 U}{N} \sum_{g=1,2} n_g \left(n_g - 1\right) + A \sin(\Omega t) \left(n_2 - n_1\right),
	\end{gathered}
\end{equation}
где первое слагаемое отвечает за перемещение бозонов между двумя узлами димера, второе слагаемое "--- за взаимодействие бозонов с силой \(U\), находящихся в  одном и том же узле и третье слагаемое "--- периодическая модуляция с периодом $T$. 
Операторы \(b_j\) и \(b^\dagger_j\) соответствуют рождению и уничтожению бозона на сайте \(j\), \(n_j = b^\dagger_j b_j\) "--- оператор числа частиц на сайте \(j\).
Каждое базисное состояние системы определяется числом бозонов в первом сайте димера (размерность гильбертова пространства \(S = N + 1\)).
Взаимодействие с окружающей средой в уравнении \cref{eq:GKSL_base} определяется диссипативным оператором:
\begin{equation}
	\label{eq:dimer_diss}
	\begin{gathered}
		V = ( b^\dagger_1 + b^\dagger_2) \left( b_1 - b_2 \right),
	\end{gathered}
\end{equation}
который синхронизирует динамику на сайтах димера за счёт рециркуляции антисимметричных противофазных состояний в симметричные и синфазные.





\FloatBarrier
\pdfbookmark{Заключение}{conclusion}                                  % Закладка pdf
В \underline{\textbf{заключении}} приведены основные результаты работы, которые заключаются в следующем:
%% Согласно ГОСТ Р 7.0.11-2011:
%% 5.3.3 В заключении диссертации излагают итоги выполненного исследования, рекомендации, перспективы дальнейшей разработки темы.
%% 9.2.3 В заключении автореферата диссертации излагают итоги данного исследования, рекомендации и перспективы дальнейшей разработки темы.
\begin{enumerate}[beginpenalty=10000] % https://tex.stackexchange.com/a/476052/104425
	\item Разработан программный комплекс \cite{prog1}, осуществляющий численное моделирование открытых квантовых систем с большим числом состояний, включающий в себя возможность анализа отдельных квантовых траекторий, поиск асимптотических состояний системы путём численного интегрирования (при наличии модуляции в системе) или поиска собственного состояния системы.
	\item Обнаружено явление Андерсоновской локализации в открытых квантовых системах. Диссипация может быть использована для создания нетривиальных устойчивых состояний, в которых доминируют несколько локализованных андерсоновских мод пространственно-неоднородного гамильтониана \cite{Yusipov2017}.
	\item Был найден механизм управления асимптотическим состоянием в открытой квантовой системе с локализацией Андерсона. Данное состояние может быть локализовано в любом месте спектра гамильтониана, за счёт управляемой синтетической диссипации. Полученные таким образом состояния являются устойчивыми к дефазирующей диссипации \cite{Vershinina2017}.
	\item Были изучены различные типы распространения волновых пакетов открытой квантовой системе с локализацией Андерсона. В частности, баллистический режим вызванный суммарным взаимодействием беспорядка и диссипации \cite{Yusipov2018}. 
	\item Было обнаружено явление многочастичной локализации в открытых квантовых системах.
	Нетривиальная диссипация, действующая на соседних сайтах решётки приводит к корреляциям между далеко расположенными сайтами, и  даже достаточно слабая дефазирующая не может их разрушить. 
	В то же время механизмы многочастичной локализации, индуцированные гамильтонианом, пытаются ограничить корреляции длиной локализации. Были предложены новые количественные идентификаторы многочастичной локализации в открытых квантовых системах \cite{Vakulchyk2018}.
	\item Обнаружен квантовый аналог бифуркации Неймарка"--~Сакера (рождение тора из-за неустойчивости предельного цикла) в модели открытого периодически модулируемого димера \cite{Yusipov2019_1}.
	\item Разработан и реализован алгоритм нахождения старшего квантового показателя Ляпунова, основанный на методе квантовых траекторий. Данная разработка позволяет выявить сложную структуру регулярных и хаотических областей, дать количественную оценку диссипативному квантовому хаосу. Как и классический старший показатель Ляпунова, его квантовый аналог становится положительным в случае хаотической динамики системы \cite{Yusipov2019_2}.
	\item Предложен экспериментально осуществимый подход к обнаружению регулярных и хаотических режимов в определённом классе открытых квантовых систем на основе анализа статистики распределения времени ожидания фотона. При переходе в режим квантового хаоса режиме появляется степенная асимптотика в распределении времени ожидания фотона \cite{Yusipov2020}.
\end{enumerate}

Благодаря очень быстрому прогрессу в области прикладных квантовых вычислений сверхпроводящие чипы используются теперь не только для выполнения квантовых вычислений, но и для изучения многочастичной локализации и разнообразных фаз квантовых многочастичных систем, таким образом предоставляя новый, хорошо контролируемый и настраиваемый стенд для изучения сложных квантовых систем. 

Исследование диссипативного квантового хаоса и его механизмов позволит использовать диссипативные эффекты (вместо того, чтобы бороться и подавлять их) для создания принципиально новых режимов сверхпроводящих квантовых систем, решающих задачу устойчивой обработки квантовой информации на длительных временных масштабах.


\pdfbookmark{Литература}{bibliography}                                % Закладка pdf

\ifdefmacro{\microtypesetup}{\microtypesetup{protrusion=false}}{} % не рекомендуется применять пакет микротипографики к автоматически генерируемому списку литературы
\urlstyle{rm}                               % ссылки URL обычным шрифтом
\ifnumequal{\value{bibliosel}}{0}{% Встроенная реализация с загрузкой файла через движок bibtex8
  \renewcommand{\bibname}{\large \bibtitleauthor}
  \nocite{*}
  \insertbiblioauthor           % Подключаем Bib-базы
  %\insertbiblioexternal   % !!! bibtex не умеет работать с несколькими библиографиями !!!
}{% Реализация пакетом biblatex через движок biber
  % Цитирования.
  %  * Порядок перечисления определяет порядок в библиографии (только внутри подраздела, если `\insertbiblioauthorgrouped`).
  %  * Если не соблюдать порядок "как для \printbibliography", нумерация в `\insertbiblioauthor` будет кривой.
  %  * Если цитировать каждый источник отдельной командой --- найти некоторые ошибки будет проще.
  %
  %% authorvak
  %\nocite{vakbib1}%
  %\nocite{vakbib2}%
  %
  %% authorwos
  \nocite{Yusipov2017}%
  \nocite{Vershinina2017}%
  \nocite{Yusipov2018}%
  \nocite{Vakulchyk2018}%
  \nocite{Yusipov2019_2}%
  \nocite{Yusipov2019_1}%
  \nocite{Yusipov2020}%
  \nocite{Liniov2019}%
  %
  %% authorscopus
  %\nocite{scbib1}%
  %
  %% authorpathent
  %\nocite{patbib1}%
  %
  %% authorprogram
  \nocite{prog1}%
  %
  %% authorconf
  \nocite{sessiann_2017}%
  \nocite{rf_2017}%
  \nocite{sfy_2017}%
  \nocite{shilnikov_2017}%
  \nocite{sessiann_2018}%
  \nocite{rf_2018}%
  \nocite{nnnph_2018}%
  \nocite{physcon_2019}%
  \nocite{rf_2020}%
  %\nocite{confbib1}%
  %\nocite{confbib2}%
  %
  %% authorother
  %\nocite{bib1}%
  %\nocite{bib2}%

  \ifnumgreater{\value{usefootcite}}{0}{
    \begin{refcontext}[labelprefix={}]
      \ifnum \value{bibgrouped}>0
        \insertbiblioauthorgrouped    % Вывод всех работ автора, сгруппированных по источникам
      \else
        \insertbiblioauthor      % Вывод всех работ автора
      \fi
    \end{refcontext}
  }{
  \ifnum \totvalue{citeexternal}>0
    \begin{refcontext}[labelprefix=A]
      \ifnum \value{bibgrouped}>0
        \insertbiblioauthorgrouped    % Вывод всех работ автора, сгруппированных по источникам
      \else
        \insertbiblioauthor      % Вывод всех работ автора
      \fi
    \end{refcontext}
  \else
    \ifnum \value{bibgrouped}>0
      \insertbiblioauthorgrouped    % Вывод всех работ автора, сгруппированных по источникам
    \else
      \insertbiblioauthor      % Вывод всех работ автора
    \fi
  \fi
  %  \insertbiblioauthorimportant  % Вывод наиболее значимых работ автора (определяется в файле characteristic во второй section)
  \begin{refcontext}[labelprefix={}]
      \insertbiblioexternal            % Вывод списка литературы, на которую ссылались в тексте автореферата
  \end{refcontext}
  % Невидимый библиографический список для подсчёта количества внешних публикаций
  % Используется, чтобы убрать приставку "А" у работ автора, если в автореферате нет
  % цитирований внешних источников.
  \printbibliography[heading=nobibheading, section=0, env=countexternal, keyword=biblioexternal, resetnumbers=true]%
  }
}
\ifdefmacro{\microtypesetup}{\microtypesetup{protrusion=true}}{}
\urlstyle{tt}                               % возвращаем установки шрифта ссылок URL
