\chapter*{Словарь терминов}             % Заголовок
\addcontentsline{toc}{chapter}{Словарь терминов}  % Добавляем его в оглавление

\textbf{Квантовый компьютер} : вычислительное устройство, осуществляющее эффективное решение задач широкого спектра за счет использования квантомеханических эффектов.

\textbf{Кубит} : элементарная единица информации в квантововм компьютере.

\textbf{Запутанность} : квантомеханический эффект, характеризующийся наличием корреляций между квантовыми частицами, которые не могут быть описаны в рамках классической статистики.

\textbf{Диссипация} : процесс взаимодействия квантовой системы с её окружением.

\textbf{Локализация Андерсона} : явление отсутствия диффузии волн в неупорядоченной среде.

\textbf{Термализация} : процесс достижения физическими телами теплового равновесия посредством взаимного взаимодействия (состояние c равнораспределённой энергией и однородной температурой, которое максимизирует энтропию системы). 

\textbf{Многочастичная локализация (MBL)} : явление, происходящее в квантовых системах многих тел и характеризующееся тем, что система не может быть термализована и сохраняет память о своем начальном состоянии в локальных наблюдаемых.

\textbf{Бифуркация Неймарка"--~Сакера} : рождение тора (инвариантная кривая в сечении Пуанкаре) из-за неустойчивости предельного цикла (неподвижная точки отображения Пуанкаре)


