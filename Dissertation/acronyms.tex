\chapter*{Список сокращений и условных обозначений} % Заголовок
\addcontentsline{toc}{chapter}{Список сокращений и условных обозначений}  % Добавляем его в оглавление
% при наличии уравнений в левой колонке значение параметра leftmargin приходится подбирать вручную
\begin{description}[align=right,leftmargin=3.5cm]
\item[MBL] many-body localization, многочастичная локализация
\item[\(\rho\)] матрица плотности
\item[\(\rho^A\)] асимптотическая матрица плотности
\item[\(H\)] гамильтониан квантовой системы
\item[\(S\)] число состояний в системе
\item[\(\{V_k\}\)] множество диссипативных операторов
\item[\(\gamma_k\)] скорость диссипации \(k\)-го канала
\item[\(K\)] количество каналов диссипации
\item[\(\alpha\)] фаза неэрмитовых диссипативных операторов
\item[\(L\)] линдбладиан, матрица оператора Линдблада
\item[\(\idmtx\)] единичная матрица
\item[\(N\)] число состояний в квантовой системе
\item[\(t^C\)] конкретное время наблюдения
\item[\(t^A\)] время, необходимое системе для достижения асимптотического состояния равновесия
\item[\(t^O\)] время наблюдения за системой, после достижения асимптотического состояния равновесия
\item[\(| \psi_j (t) \rangle\)] волновая функция, вектор состояния \(j\)-ой траектории в момент времени \(t\)
\item[\(M_r\)] количество квантовых траекторий
\item[\(b_n\)] оператор рождения квантовой частицы на \(n\)-ом сайте решётки
\item[\(b^\dagger_n\)] оператор уничтожения квантовой частицы на \(n\)-ом сайте решётки
\item[\(N_r\)] количество реализаций беспорядка
\item[PDF] probability densidy function, функция распределения плотности вероятностей
\item[OSEE, \(S^\natural\)] operator"--~space entanglement entropy, энтропия запутанности операторного пространства
\item[RCLS,  \(r\)] ratio of consecutive level spacing, соотношение последовательных уровней
\item[GOE] гауссов ортогональный ансамбль случайных матриц
\item[GUE] гауссов унитарный ансамбль случайных матриц
\end{description}
