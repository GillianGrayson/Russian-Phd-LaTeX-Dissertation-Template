\chapter{Хаос в открытых квантовых системах}\label{ch:ch3}

Взаимосвязь между квантовыми системами и их классическими аналогами (в частности, среднеполевые приближения) являются ключевой проблемой теории квантового хаоса \cite{Stockmann2006}. 
До настоящего времени данная связь анализировалась в основном с точки зрения спектральных характеристик квантовых гамильтонианов. 
В данном контексте одной из самых важных вех в развитии квантового хаоса являлось установление связи между спектральной статистикой и переходами от регулярной динамики к хаотической (и наоборот) в фазовом пространстве соответствующих классических систем \cite{Stockmann2006}.

Бифуркационный анализ \cite{Poincare1885} является одним из основных подходов изучения нелинейной динамики и ее приложений \cite{Kuznetsov2004}.
Применение бифуркационного анализа в сфере квантовой физики долгое время рассматривалось только в контексте изолированных систем  "---  гамильтонов хаос, спектральные характеристики которого в квантовых системах к настоящему времени являются очень хорошо изученными \cite{Casati1979, Gutzwiller1990}.
Квантовые следы бифуркаций, то есть существенных изменений в структуре фазового пространства классических гамильтоновых систем при небольшом изменении параметра(ов), также исследовались в работах \cite{Hines2005, Santos2006, Nemes2006}. Здесь было обнаружено, что классические бифуркации типа «вилка» и Андронова"--~Хопфа \cite{Wiggins2013} в среднеполевых моделях связаны с резкими изменениями запутанности основного состояния в соответствующих квантовых моделях.
Кроме этого, в работе \cite{Zibold2010} было обнаружено, что бифуркация типа «вилка» соответствует переходу от динамики Раби к динамике Джозефсона в экспериментах с рубидиевым конденсатом Бозе"--~Эйнштейна.

Дальнейшее применение бифуркационного анализа к открытым квантовым системам в связке с соответствующими среднеполевыми моделями будет способствовать развитию теории диссипативного квантового хаоса. 
В случае открытых систем бифуркации могут иметь более серьёзные последствия, чем в гамильтоновом случае, потому что они будут влиять на стационарное состояние системы, а не только на конкретное собственное состояние изолированной системы.
Однако, существуют значительные трудности в построении среднеполевых приближений для открытых квантовых систем.

В недавней работе \cite{Ivanchenko2017} было обобщено понятие бифуркации для случая квантовых диссипативных систем.
Обычно квантовые бифуркации визуализируются путём вычисления квазиклассических распределений фазового пространства, типа Хусими или Вигнера \cite{Stockmann2006}, структурные изменения которых с изменением параметра воспроизводят бифуркации в классическом фазовом пространстве. 
Например, квантовая бифуркация удвоения периода рассматривается как переход от унимодального к бимодальному распределению Хусими \cite{Hartmann2017, Wang2018}.
В работе \cite{Ivanchenko2017}, используя многочастичную модель квантового димера было показано, что асимптотическая матрица плотности открытой системы может быть использована для построения бифуркационной диаграммы классического типа, которая может быть связана с классической динамикой в среднеполевом приближении рассматриваемой квантовой модели.
Такой подход позволил преодолеть технические ограничения при вычислении распределения Хусими для систем с большим числом состояний.
В итоге были обнаружены квантовые аналоги некоторых классических типов бифуркаций "--- седлоузловая, типа «вилки», а также сценарий перехода к квантовому хаосу через каскад бифуркаций удвоения периода.

Другим важным направлением развития теории диссипативного квантового хаоса является создание инструментов для его количественной оценки.
Одна из основополагающих концепций теории динамического хаоса заключается в том, что сложная хаотическая динамика возникает из-за локальной нестабильности, которая заставляет две изначально близкие траектории расходиться. 
Это расхождение обычно количественно оценивается с помощью показателей Ляпунова "--- мощного инструмента для количественной оценки динамического хаоса.
Попытки обобщения показателя Ляпунова на квантовую динамику предпринимаются давно. 
В большинстве исследований рассматриваются случаи гамильтоновых систем, где впервые была создана спектральная теория квантового хаоса \cite{Haake2018}.
Соответствующие обобщения варьируются от ранних идей использования функций квазивероятности и определения квантовых показателей Ляпунова в терминах «расстояния» между ними \cite{Toda1987, Haake1992, Manko2000} до самых недавних достижений, основанных на вневременных корреляционных функциях \cite{Rozenbaum2017, Liao2018, ChavezCarlos2019}.

Быстрый прогресс в экспериментальной квантовой физике, особенно в таких областях, как квантовые электродинамические (КЭД) системы \cite{Walther2006}, квантовые оптических системы \cite{Aspelmeyer2014} и поляритонные устройства\cite{Feurer2003}, способствовал переходу к более реалистичному негамильтонову описанию квантовых систем. 
Динамика таких открытых систем существенно диссипативна и не менее сложен и универсальна, чем унитарная \cite{Diehl2008, Budich2015}. 
Существует множество доказательств, как вычислительных, так и экспериментальных, что асимптотические состояния открытых квантовых систем могут давать (при измерении, например, с помощью квантовой томографии) структуры, подобные к классическим хаотическим аттракторам \cite{Spiller1994, Brun1996, Hartmann2017, Ivanchenko2017, Carlo2017, Wang2018}. 
Подходы, пытающиеся согласовать свойства спектров генераторов диссипативной квантовой эволюции (линдбладианов) \cite{book2007} или спектров асимптотической матрицы плотности \cite{Hartmann2017, Ivanchenko2017, Prosen2013} с переходами между регулярным и хаотическим режимами в соответствующих среднеполевых моделях, дали некоторые интересные результаты. Однако, область количественной оценки диссипативного квантового хаоса все ещё остаётся малоизученной.

Также отдельный интерес вызывают не только теоретические характеристики квантового диссипативного хаоса, но и квантификаторы динамики системы, которые можно получить из реального физического эксперимента. 
В качестве модели в данном контексте можно рассматривать  квантовый электродинамические (КЭД) резонатор, в экспериментальной реализации которого можно наблюдать за эмиссией фотонов \cite{Walther2006, Arakawa2015}.
Последние достижения в данной области позволяют, например, изготовить полупроводниковую квантовую точку и встроить ее в микрополость \cite{Arakawa2015}. 
Квантовая точка может иметь от двух до четырёх уровней энергии, a взаимодействие между модами резонатора и точечными экситонами также может быть настроено \cite{Reithmaier2004, Hennessy2007}. 
Небольшое количество энергетических уровней делает квантовые точки хорошими кандидатами для реализации кубитов, в то время как взаимодействие между ними может регулироваться при помощи резонатора.

В разделе \cref{sec:ch3/dimer} описывается модель взаимодействующих бозонов в открытом квантовом димере с периодической модуляцией.

\section{Открытый квантовый димер}\label{sec:ch3/dimer}
Модель состоит из \(N\) неразличимых взаимодействующих бозонов, которые перемещаются между двумя узлами периодически модулируемого димера. 
Данная модель является популярной в теоретических исследованиях \cite{Vardi2001, Trimborn2008, Poletti2012} и имеет экспериментальную реализацию \cite{Gross2010, Tomkovic2017}. Кроме этого, в данной модели были обнаружены различные хаотические и регулярные режимы \cite{Hartmann2017, Ivanchenko2017, Carlo2017, Wang2018}. 

\subsection{Квантовая модель}\label{subsec:ch3/dimer/quantum}

Унитарная часть уравнения Линдблада \labelcref{eq:GKSL_base, eq:GKSL_lindbladian} задается следующим гамильтонианом:
\begin{equation}
	\label{eq:dimer_H}
	\begin{gathered}
		H(t) = -\mathcal{J} \left(b^\dagger_1 b_2 + b^\dagger_2 b_1\right) + \frac{2 U}{N} \sum_{g=1,2} n_g \left(n_g - 1\right) + \varepsilon(t) \left(n_2 - n_1\right),
	\end{gathered}
\end{equation}
где первое слагаемое отвечает за перемещение бозонов между двумя узлами димера с коэффициентом \(\mathcal{J}\), второе слагаемое "--- за взаимодействие бозонов с силой \(U\), находящихся в  одном и том же узле и третье слагаемое "--- периодическая модуляция с функцией \(\varepsilon(t)\). 
Операторы \(b_j\) и \(b^\dagger_j\) соответствуют рождению и уничтожению бозона на сайте \(j\), \(n_j = b^\dagger_j b_j\) "--- оператор числа частиц на сайте \(j\).

В дальнейшем будут рассмотрены два типа периодической модуляции димера.
Непрерывная (C - Continuous) функция модуляции:
\begin{equation}
	\label{eq:dimer_mod_c}
	\begin{gathered}
		\varepsilon_{C}(t) = \varepsilon_{C}(t + T) = A \sin(\Omega t),
	\end{gathered}
\end{equation}
где \(T\) "--- период модуляции, \(A\) "--- амплитуда модуляции (разность энергий между сайтами димера), \(\Omega = 2 \pi / T\).
Второй тип модуляции "--- кусочно-постоянная (P - Piecewise):
\begin{equation}
	\label{eq:dimer_mod_p}
	\begin{gathered}
		\varepsilon_{P}(t) = \varepsilon_{P}(t + T) = \mu_0 + \mu_1 Q(t), \\
		Q(\tau) = 1, \quad 0 < \tau \le T/2, \\
		Q(\tau) = 0, \quad T/2 < \tau \le T,
	\end{gathered}
\end{equation}
где \(T\) "--- период модуляции, \(\mu_0\) и \(\mu_1\) "--- постоянная и переменная разница в уровне энергий между сайтами димера, \(Q(\tau)\) "--- периодическая функция, принимающая два значения.

Гильбертово пространство системы имеет размерность \(S = N + 1\) и каждое базисное состояние системы представляется можно обозначить как число бозонов в первом сайте димера \( \lbrace \left|n\right>\rbrace\), где \(n = 0,\ldots,N\). Таким образом, размер системы задаётся количеством бозонов в димере.

Во второй части уравнения Линдбдлада \labelcref{eq:GKSL_base, eq:GKSL_lindbladian}, отвечающей за взаимодействие с окружающей средой, участвует только один эксперементально-релевантный \cite{Diehl2008, Kraus2008} диссипативный оператор (\(K=1\)), который действует между двумя сайтами димера. Данный диссипатор имеет вид:
\begin{equation}
	\label{eq:dimer_diss}
	\begin{gathered}
		V = ( b^\dagger_1 + b^\dagger_2) \left( b_1 - b_2 \right),
	\end{gathered}
\end{equation}
и синхронизирует динамику на сайтах димера за счёт рециркуляции антисимметричных противофазных состояний в симметричные и синфазные. Скорость диссипации \(\gamma\) является независимой от времени величиной.
При использовании метода квантовых траекторий \cref{sec:ch1/qj} для каждой траектории \(\left| \psi_j(t) \right\rangle\) ( \(j\) - индекс) в момент времени \(t\) будут вычисляться следующие значения наблюдаемых "--- нормированного числа частиц в левом сайте димера:
\begin{equation}
	\label{eq:dimer_num_bosons}
	\begin{gathered}
		n_j(t) =  \frac{1}{N}\langle \psi_j(t)| b^\dagger_1 b_1 | \psi_j(t) \rangle,
	\end{gathered}
\end{equation}
и нормированной энергии:
\begin{equation}
	\label{eq:dimer_energy}
	\begin{gathered}
		e_j(t) = \frac{1}{N} \langle \psi_j(t)| H(t) | \psi_j(t) \rangle.
	\end{gathered}
\end{equation}

\subsection{Среднеполевое приближение}\label{subsec:ch3/dimer/meanfield}
В пределе бесконечного числа частиц \(N \to \infty\) в среднеполевом приближении модели квантового димера используются следующие спиновые операторы:
\begin{equation}
	\label{eq:dimer_meanfield_spin}
	\begin{gathered}
		\mathcal{S}_x = \frac{1}{2 N} \left(b^\dagger_1 b_2 + b^\dagger_2 b_1\right), \\
		\mathcal{S}_y = - \frac{i}{2 N} \left(b^\dagger_1 b_2 - b^\dagger_2 b_1\right), \\
		\mathcal{S}_z = \frac{1}{2 N} \left(b^\dagger_1 b_1 - b^\dagger_2 b_2\right), \\
	\end{gathered}
\end{equation}
чья эволюция рассматривается в представлении Гейзенберга \cite{Breuer2007}. 
В пределе бесконечного числа частиц коммутатором первых двух спиновых операторов можно пренебречь: \(\left[\mathcal{S}_x, \mathcal{S}_y\right] = i \mathcal{S}_z \stackrel{N\to\infty}{=} 0\) (так как является бесконечно малой порядка $\mathcal{O}(N^{-1})$). 
Это же свойство справедливо для всех остальных коммутаторов, полученных из разных перестановок спиновых операторов.
Заменяя спиновые операторы их математическим ожиданием \(\mathscr{S}_k = tr\left[\rho\mathcal{S}_k\right]\), можно получить следующую систему дифференциальных уравнений \cite{Hartmann2017}:
\begin{equation}
	\label{eq:dimer_meanfield_ode_spin}
		\left\{
		  \begin{array}{rl}
		    \frac{d \mathscr{S}_x}{dt} = & 2\varepsilon (t) \mathscr{S}_y - 8 U \mathscr{S}_z \mathscr{S}_y + 8 \gamma \left(\mathscr{S}_y^2+\mathscr{S}_z^2\right) \\
		    \frac{d \mathscr{S}_y}{dt} = & -2\varepsilon \left(t\right) \mathscr{S}_x + 8 U \mathscr{S}_x \mathscr{S}_z - 2 \mathcal{J} \mathscr{S}_z + 8\gamma \mathscr{S}_x \mathscr{S}_y \\
		    \frac{d \mathscr{S}_z}{dt} = & -2 \mathcal{J} \mathscr{S}_y - 8 \gamma \mathscr{S}_x \mathscr{S}_z
		  \end{array}
		\right.
\end{equation}	
В данной системе есть интеграл движения: \(\mathscr{S}^2 = \mathscr{S}_x^2 + \mathscr{S}_y^2 + \mathscr{S}_z^2\) и эволюция среднеполевой модели ограничена поверхностью сферы Блоха \cite{Nielsen2010}:
\begin{equation}
	\label{eq:dimer_meanfield_sphere}
	\left\{
	\begin{array}{rl}
		\mathscr{S}_x = & \frac{1}{2} \cos{\left(\varphi\right)}\sin{\left(\vartheta\right)}\\
		\mathscr{S}_y = & \frac{1}{2} \sin{\left(\varphi\right)}\sin{\left(\vartheta\right)}\\
		\mathscr{S}_z = & \frac{1}{2} \cos{\left(\vartheta\right)}
	\end{array}
	\right.
\end{equation}
Система дифференциальных, описывающих движение на данной сфере выглядит следующим образом:
\begin{equation}
	\label{eq:dimer_meanfield_ode_sphere}
	\left\{
	\begin{array}{rl}
		\dot{\vartheta} = & -2\mathcal{J}\sin{\left(\varphi\right)} + 4\gamma \cos{\left(\varphi\right)}\cos{\left(\vartheta\right)} \\
		\dot{\varphi} = & -2 \mathcal{J} \frac{\cos{\left(\vartheta\right)}}{\sin{\left(\vartheta\right)}} - 2\varepsilon\left(t\right) + 4U \cos{\left(\vartheta\right)} - 4\gamma\frac{\sin{\left(\varphi\right)}}{\sin{\left(\vartheta\right)}}
	\end{array}
	\right.
\end{equation}
Число бозонов в первом сайте димера вычисляется по формуле:
\begin{equation}
	\label{eq:dimer_meanfield_num_bosons}
	\begin{gathered}
		n=\frac{N}{2}(1+\cos{\left(\vartheta\right)}).
	\end{gathered}
\end{equation}
Данная классическая нелинейная система \cref{eq:dimer_meanfield_ode_sphere} будет играть опорную роль и результаты полученные для неё будут сравниваться с результатами квантовой системы, описанной в разделе \cref{subsec:ch3/dimer/quantum}.


%\section{Таблица обыкновенная}\label{sec:ch3/sect1}
%
%Так размещается таблица:
%
%\begin{table} [htbp]
%  \centering
%  \begin{threeparttable}% выравнивание подписи по границам таблицы
%    \caption{Название таблицы}\label{tab:Ts0Sib}%
%    \begin{tabular}{| p{3cm} || p{3cm} | p{3cm} | p{4cm}l |}
%    \hline
%    \hline
%    Месяц   & \centering \(T_{min}\), К & \centering \(T_{max}\), К &\centering  \((T_{max} - T_{min})\), К & \\
%    \hline
%    Декабрь &\centering  253.575   &\centering  257.778    &\centering      4.203  &   \\
%    Январь  &\centering  262.431   &\centering  263.214    &\centering      0.783  &   \\
%    Февраль &\centering  261.184   &\centering  260.381    &\centering     \(-\)0.803  &   \\
%    \hline
%    \hline
%    \end{tabular}
%  \end{threeparttable}
%\end{table}
%
%\begin{table} [htbp]% Пример записи таблицы с номером, но без отображаемого наименования
%  \centering
%  \begin{threeparttable}% выравнивание подписи по границам таблицы
%    \caption{}%
%    \label{tab:test1}%
%    \begin{SingleSpace}
%      \begin{tabular}{| c | c | c | c |}
%        \hline
%        Оконная функция & \({2N}\)& \({4N}\)& \({8N}\)\\ \hline
%        Прямоугольное   & 8.72  & 8.77  & 8.77  \\ \hline
%        Ханна           & 7.96  & 7.93  & 7.93  \\ \hline
%        Хэмминга        & 8.72  & 8.77  & 8.77  \\ \hline
%        Блэкмана        & 8.72  & 8.77  & 8.77  \\ \hline
%      \end{tabular}%
%    \end{SingleSpace}
%  \end{threeparttable}
%\end{table}
%
%Таблица~\cref{tab:test2} "--- пример таблицы, оформленной в~классическом книжном
%варианте или~очень близко к~нему. \mbox{ГОСТу} по~сути не~противоречит. Можно
%ещё~улучшить представление, с~помощью пакета \verb|siunitx| или~подобного.
%
%\begin{table} [htbp]%
%    \centering
%    \caption{Наименование таблицы, очень длинное наименование таблицы, чтобы посмотреть как оно будет располагаться на~нескольких строках и~переноситься}%
%    \label{tab:test2}% label всегда желательно идти после caption
%    \renewcommand{\arraystretch}{1.5}%% Увеличение расстояния между рядами, для улучшения восприятия.
%    \begin{SingleSpace}
%        \begin{tabular}{@{}@{\extracolsep{20pt}}llll@{}} %Вертикальные полосы не используются принципиально, как и лишние горизонтальные (допускается по ГОСТ 2.105 пункт 4.4.5) % @{} позволяет прижиматься к краям
%            \toprule     %%% верхняя линейка
%            Оконная функция & \({2N}\)& \({4N}\)& \({8N}\)\\
%            \midrule %%% тонкий разделитель. Отделяет названия столбцов. Обязателен по ГОСТ 2.105 пункт 4.4.5
%            Прямоугольное   & 8.72  & 8.77  & 8.77  \\
%            Ханна           & 7.96  & 7.93  & 7.93  \\
%            Хэмминга        & 8.72  & 8.77  & 8.77  \\
%            Блэкмана        & 8.72  & 8.77  & 8.77  \\
%            \bottomrule %%% нижняя линейка
%        \end{tabular}%
%    \end{SingleSpace}
%\end{table}
%
%\section{Таблица с многострочными ячейками и примечанием}
%
%В таблице \cref{tab:makecell} приведён пример использования команды
%\verb+\multicolumn+ для объединения горизонтальных ячеек таблицы,
%и команд пакета \textit{makecell} для добавления разрыва строки внутри ячеек.
%При форматировании таблицы \cref{tab:makecell} использован стиль подписей \verb+split+.
%Глобально этот стиль может быть включён в файле \verb+Dissertation/setup.tex+ для диссертации и в
%файле \verb+Synopsis/setup.tex+ для автореферата.
%Однако такое оформление не~соответствует ГОСТ.
%
%\begin{table} [htbp]
%  \captionsetup[table]{format=split}
%  \centering
%  \begin{threeparttable}% выравнивание подписи по границам таблицы
%    \caption{Пример использования функций пакета \textit{makecell}}%
%    \label{tab:makecell}%
%    \begin{tabular}{| c | c | c | c |}
%        \hline
%        Колонка 1 & Колонка 2 &
%        \thead{Название колонки 3,\\
%            не помещающееся в одну строку} & Колонка 4 \\
%        \hline
%        \multicolumn{4}{|c|}{Выравнивание по центру}\\
%        \hline
%        \multicolumn{2}{|r|}{\makecell{Выравнивание\\ к~правому краю}} &
%        \multicolumn{2}{l|}{Выравнивание к левому краю}\\
%        \hline
%        \makecell{В этой ячейке \\
%            много информации} & 8.72 & 8.55 & 8.44\\
%        \cline{3-4}
%        А в этой мало         & 8.22 & \multicolumn{2}{c|}{5}\\
%        \hline
%    \end{tabular}%
%  \end{threeparttable}
%\end{table}
%
%Таблицы~\cref{tab:test3,tab:test4} "--- пример реализации расположения
%примечания в~соответствии с ГОСТ 2.105. Каждый вариант со своими достоинствами
%и~недостатками. Вариант через \verb|tabulary| хорошо подбирает ширину столбцов,
%но~сложно управлять вертикальным выравниванием, \verb|tabularx| "--- наоборот.
%\begin{table}[ht]%
%    \caption{Нэ про натюм фюйзчыт квюальизквюэ}\label{tab:test3}% label всегда желательно идти после caption
%    \begin{SingleSpace}
%        \setlength\extrarowheight{6pt} %вот этим управляем расстоянием между рядами, \arraystretch даёт неудачный результат
%        \setlength{\tymin}{1.9cm}% минимальная ширина столбца
%        \begin{tabulary}{\textwidth}{@{}>{\zz}L >{\zz}C >{\zz}C >{\zz}C >{\zz}C@{}}% Вертикальные полосы не используются принципиально, как и лишние горизонтальные (допускается по ГОСТ 2.105 пункт 4.4.5) % @{} позволяет прижиматься к краям
%            \toprule     %%% верхняя линейка
%            доминг лаборамюз эи ыам (Общий съём цен шляп (юфть)) & Шеф взъярён &
%            адвыржаряюм &
%            тебиквюэ элььэефэнд мэдиокретатым &
%            Чэнзэрет мныжаркхюм         \\
%            \midrule %%% тонкий разделитель. Отделяет названия столбцов. Обязателен по ГОСТ 2.105 пункт 4.4.5
%            Эй, жлоб! Где туз? Прячь юных съёмщиц в~шкаф Плюш изъят. Бьём чуждый цен хвощ! &
%            \({\approx}\) &
%            \({\approx}\) &
%            \({\approx}\) &
%            \( + \) \\
%            Эх, чужак! Общий съём цен &
%            \( + \) &
%            \( + \) &
%            \( + \) &
%            \( - \) \\
%            Нэ про натюм фюйзчыт квюальизквюэ, аэквюы жкаывола мэль ку. Ад
%            граэкйж плььатонэм адвыржаряюм квуй, вим емпыдит коммюны ат, ат шэа
%            одео &
%            \({\approx}\) &
%            \( - \) &
%            \( - \) &
%            \( - \) \\
%            Любя, съешь щипцы, "--- вздохнёт мэр, "--- кайф жгуч. &
%            \( - \) &
%            \( + \) &
%            \( + \) &
%            \({\approx}\) \\
%            Нэ про натюм фюйзчыт квюальизквюэ, аэквюы жкаывола мэль ку. Ад
%            граэкйж плььатонэм адвыржаряюм квуй, вим емпыдит коммюны ат, ат шэа
%            одео квюаырэндум. Вёртюты ажжынтиор эффикеэнди эож нэ. &
%            \( + \) &
%            \( - \) &
%            \({\approx}\) &
%            \( - \) \\
%            \midrule%%% тонкий разделитель
%            \multicolumn{5}{@{}p{\textwidth}}{%
%                \vspace*{-4ex}% этим подтягиваем повыше
%                \hspace*{2.5em}% абзацный отступ - требование ГОСТ 2.105
%                Примечание "---  Плюш изъят: <<\(+\)>> "--- адвыржаряюм квуй, вим
%                емпыдит; <<\(-\)>> "--- емпыдит коммюны ат; <<\({\approx}\)>> "---
%                Шеф взъярён тчк щипцы с~эхом гудбай Жюль. Эй, жлоб! Где туз?
%                Прячь юных съёмщиц в~шкаф. Экс-граф?
%            }
%            \\
%            \bottomrule %%% нижняя линейка
%        \end{tabulary}%
%    \end{SingleSpace}
%\end{table}
%
%Если таблица~\cref{tab:test3} не помещается на той же странице, всё
%её~содержимое переносится на~следующую, ближайшую, а~этот текст идёт перед ней.
%\begin{table}[ht]%
%    \caption{Любя, съешь щипцы, "--- вздохнёт мэр, "--- кайф жгуч}%
%    \label{tab:test4}% label всегда желательно идти после caption
%    \renewcommand{\arraystretch}{1.6}%% Увеличение расстояния между рядами, для улучшения восприятия.
%    \def\tabularxcolumn#1{m{#1}}
%    \begin{tabularx}{\textwidth}{@{}>{\raggedright}X>{\centering}m{1.9cm} >{\centering}m{1.9cm} >{\centering}m{1.9cm} >{\centering\arraybackslash}m{1.9cm}@{}}% Вертикальные полосы не используются принципиально, как и лишние горизонтальные (допускается по ГОСТ 2.105 пункт 4.4.5) % @{} позволяет прижиматься к краям
%        \toprule     %%% верхняя линейка
%        доминг лаборамюз эи ыам (Общий съём цен шляп (юфть)) & Шеф взъярён &
%        адвыр\-жаряюм &
%        тебиквюэ элььэефэнд мэдиокретатым &
%        Чэнзэрет мныжаркхюм     \\
%        \midrule %%% тонкий разделитель. Отделяет названия столбцов. Обязателен по ГОСТ 2.105 пункт 4.4.5
%        Эй, жлоб! Где туз? Прячь юных съёмщиц в~шкаф Плюш изъят.
%        Бьём чуждый цен хвощ! &
%        \({\approx}\) &
%        \({\approx}\) &
%        \({\approx}\) &
%        \( + \) \\
%        Эх, чужак! Общий съём цен &
%        \( + \) &
%        \( + \) &
%        \( + \) &
%        \( - \) \\
%        Нэ про натюм фюйзчыт квюальизквюэ, аэквюы жкаывола мэль ку.
%        Ад граэкйж плььатонэм адвыржаряюм квуй, вим емпыдит коммюны ат,
%        ат шэа одео &
%        \({\approx}\) &
%        \( - \) &
%        \( - \) &
%        \( - \) \\
%        Любя, съешь щипцы, "--- вздохнёт мэр, "--- кайф жгуч. &
%        \( - \) &
%        \( + \) &
%        \( + \) &
%        \({\approx}\) \\
%        Нэ про натюм фюйзчыт квюальизквюэ, аэквюы жкаывола мэль ку. Ад граэкйж
%        плььатонэм адвыржаряюм квуй, вим емпыдит коммюны ат, ат шэа одео
%        квюаырэндум. Вёртюты ажжынтиор эффикеэнди эож нэ. &
%        \( + \) &
%        \( - \) &
%        \({\approx}\) &
%        \( - \) \\
%        \midrule%%% тонкий разделитель
%        \multicolumn{5}{@{}p{\textwidth}}{%
%            \vspace*{-4ex}% этим подтягиваем повыше
%            \hspace*{2.5em}% абзацный отступ - требование ГОСТ 2.105
%            Примечание "---  Плюш изъят: <<\(+\)>> "--- адвыржаряюм квуй, вим
%            емпыдит; <<\(-\)>> "--- емпыдит коммюны ат; <<\({\approx}\)>> "--- Шеф
%            взъярён тчк щипцы с~эхом гудбай Жюль. Эй, жлоб! Где туз? Прячь юных
%            съёмщиц в~шкаф. Экс-граф?
%        }
%        \\
%        \bottomrule %%% нижняя линейка
%    \end{tabularx}%
%\end{table}
%
%\section{Таблицы с форматированными числами}\label{sec:ch3/formatted-numbers}
%
%В таблицах \cref{tab:S:parse,tab:S:align} представлены примеры использования опции
%форматирования чисел \texttt{S}, предоставляемой пакетом \texttt{siunitx}.
%
%\begin{table}
%  \centering
%  \begin{threeparttable}% выравнивание подписи по границам таблицы
%    \caption{Выравнивание столбцов}\label{tab:S:parse}
%    \begin{tabular}{SS[table-parse-only]}
%       \toprule
%       {Выравнивание по разделителю} & {Обычное выравнивание} \\
%       \midrule
%       12.345                        & 12.345                 \\
%       6,78                          & 6,78                   \\
%       -88.8(9)                      & -88.8(9)               \\
%       4.5e3                         & 4.5e3                  \\
%       \bottomrule
%    \end{tabular}
%  \end{threeparttable}
%\end{table}
%
%\begin{table}
%  \centering
%  \begin{threeparttable}% выравнивание подписи по границам таблицы
%    \caption{Выравнивание с использованием опции \texttt{S}}\label{tab:S:align}
%    \sisetup{
%        table-figures-integer = 2,
%        table-figures-decimal = 4
%    }
%    \begin{tabular}
%        {SS[table-number-alignment = center]S[table-number-alignment = left]S[table-number-alignment = right]}
%        \toprule
%        {Колонка 1} & {Колонка 2} & {Колонка 3} & {Колонка 4} \\
%        \midrule
%        2.3456      & 2.3456      & 2.3456      & 2.3456      \\
%        34.2345     & 34.2345     & 34.2345     & 34.2345     \\
%        56.7835     & 56.7835     & 56.7835     & 56.7835     \\
%        90.473      & 90.473      & 90.473      & 90.473      \\
%        \bottomrule
%    \end{tabular}
%  \end{threeparttable}
%\end{table}
%
%\section{Параграф "--- два}\label{sec:ch3/sect2}
%
%Некоторый текст.
%
%\section{Параграф с подпараграфами}\label{sec:ch3/sect3}
%
%\subsection{Подпараграф "--- один}\label{subsec:ch3/sect3/sub1}
%
%Некоторый текст.
%
%\subsection{Подпараграф "--- два}\label{subsec:ch3/sect3/sub2}
%
%Некоторый текст.
%
%\clearpage
