
{\actuality} 
В настоящее время диссипативные квантовые системы являются неотъемлемой частью экспериментальной и технологической реальности.
Практические реализации нано- \autocite{Poot2012} и оптомеханических систем \autocite{Aspelmeyer2014}, сверхпроводящих элементов \autocite{Clarke2008} не осуществимы в полной изоляции от окружающей среды.
Эволюция идеальных когерентных квантовых систем является простым приближением реальных физических процессов. 
Все эти системы являются открытыми и их динамика существенно диссипативна.
Диссипация в данных системах является полноправным генератором эволюции, которая не менее сложна и разнообразна, чем унитарная, генерируемая гамильтонианами.
Асимптотические состояния открытых квантовых систем определится не только гамильтоновой динамикой, но взаимодействием с окружающей средой.
Данное взаимодействие не является достаточно сильным, чтобы полностью свести динамику системы к классическому случаю \autocite{Breuer2007}.
Особенностью диссипации является то, что она способна привести систему в состояния, которые недостижимы в классическом пределе.
Например, новые топологические состояния, получаемые за счёт управляемой синтетической диссипации \autocite{Diehl2011} или «чистые» сильно запутанные состояния в многочастичных квантовых системах \autocite{Kraus2008}.

На сегодняшний день теория открытых квантовых систем является хорошо развитой областью теоретической физики с широким спектром разноплановых фундаментальных результатов и методов \autocite{Breuer2007}.
Одним из самых распространённых подходов к описанию динамики открытых квантовых систем является формализм Линдблада \autocite{Lindblad1976, Gorini1976} и метод квантовых скачков \autocite{Dalibard1992, Dum1992, Plenio1998}. Последний является популярным инструментом в сфере квантовой оптики \autocite{Carmichael1993}.
Кроме того, этот метод имеет приложения в физике холодных атомов \autocite{Diehl2011, Diehl2008, Marcuzzi2014} и активно используется в контексте квантовых электродинамических (КЭД) систем \autocite{Imamoglu1999, Walther2006, Arakawa2015}.

Асимптотическое состояние открытых квантовых систем (квантовый аттрактор) полностью определяется типом взаимодействия с окружающей средой (диссипацией). 
Например, диссипация описанная в работе \autocite{Poletti2013} приводит любую квантовую систему в тривиальное состояние с максимальной энтропией (все состояния системы становятся равновероятными).
В настоящее время доминирующей тенденцией является идея «борьбы» с декогеренцией (например, увеличение времени когерентности атомных конденсатов в оптических решётках). Однако, все большее распространение получает идея диссипативной инженерии \autocite{Diehl2008}, позволяющей создавать принципиально новые состояния многочастичных систем.

Существуют области физики открытых квантовых систем, которые не являются достаточно полно изученными. В частности, явление локализации Андерсона в пространственно-неоднородных средах \autocite{Anderson1958, Kramer1993, Evers2008, Esposito2012}, которая является очень хорошо изученным феноменом в случае невзаимодействующих частиц в когерентном гамильтоновом пределе \autocite{Segev2013, Billy2008, Roati2008, Yedjour2010, Kondov2011, Jendrzejewski2012} и гораздо менее изучена в случае взаимодействий системы с окружающей средой \autocite{Breuer2007}. Другие типы локализации, индуцированные топологией плоских зон \autocite{Derzhko2006, Bergman2008, Flach2014}, также вызваны деструктивной интерференцией, препятствующей распространению волновых пакетов. Следовательно, можно ожидать, что данные типы локализации также будут исчезать при наличии в системе диссипации.

Предположение о том, что в открытых системах диссипация должна оказывать деструктивное влияние на интерференцию и разрушать локализацию Андерсона \autocite{Genway2014}, не всегда выполнимо.
Первые доказательства того, что локализация Андерсона может существовать в диссипативных системах в асимптотическом пределе, были получены для квазиклассических и классических систем. В частности, были получены экспериментальные свидетельства локализации в случайном лазере, за счёт которой уменьшается пространственное перекрытие мод и, в результате, улучшается стабильность лазера \autocite{Stano2012, Liu2014}.

В рамках данной диссертационной работы будет продемонстрирована возможность существования асимптотических состояний в открытой квантовой системе, которые сохраняют признаки локализации Андерсона \cite{Yusipov2017}. Такое асимптотическое состояние может быть спроектировано с помощью некоторого набора параметризованных диссипативных операторов (диссипаторов) \autocite{Diehl2008}, осуществляющих отбор локализованных в пространстве андерсоновских мод.
Изменяя параметры данных диссипаторов можно контролировать структуру аттракторов путём избирательной селекции собственных состояний гамильтониана \cite{Vershinina2017}. В результате в неупорядоченной системе Андерсона асимптотическое состояние может быть локализовано в любом месте спектра гамильтониана. При этом локализация будет сохраняться в случае наличия в системе добавочной дефазирующей диссипации.

Интересным является вопрос о типе распространения волновых пакетов в рассматриваемых открытых квантовых системах с локализацией Андерсона.
В основополагающей работе о локализации за счёт беспорядка \autocite{Anderson1958} было отмечено, что взаимодействие системы с локализующим гамильтонианом с окружающей средой должно оказывать сильное влияние на результирующую динамику. 
В ранее проведённых исследованиях \autocite{Gurvitz2000, Nowak2012, Flores1999} было установлено, что декогеренция уничтожает все следы локализации и приводит систему к тривиальному режиму нормальной диффузии.
Недавние результаты, однако, демонстрируют гораздо более богатую физику: даже когда асимптотическое состояние является тривиальным равномерным распределением, процесс релаксации проявляет неоднородную динамику с признаками метастабильности \autocite{Genway2014}. А в работах \autocite{Valenti2015, Spagnolo2015, Spagnolo2016} есть указания на то, что диссипация может даже стабилизировать метастабильные состояния.

Представленные в данной диссертационной работе результаты демонстрируют, что диффузия волновых пакетов имеет место лишь для ограниченного класса диссипаторов, которые в самом деле уничтожают все признаки локализации в асимптотическом состоянии.
В случае использования других параметров диссипативных операторов, будут обнаружены нетривиальные режимы баллистического распространения волновых пакетов, механизм возникновения которых связан с конкуренцией между
диссипацией и беспорядком \cite{Yusipov2018}.

Помимо классической локализации Андерсона \autocite{Anderson1958}, можно рассмотреть ее расширение в контексте систем с большим количеством частиц "--- многочастичную локализацию (MBL) \autocite{Gornyi2005, Basko2006}.
Существует целый спектр определений и квантификаторов этого многогранного явления, нацеленных на выделение специфических свойств систем с многочастичной локализацией.
Среди них можно выделить отсутствие проводимости \autocite{Gornyi2005} (даже в пределе бесконечной температуры \autocite{Basko2006}), медленный логарифмический рост энтропии запутанности при уменьшении параметра взаимодействия между частицами \autocite{Chiara2006, Znidaric2008, Bardarson2012, Serbyn2013_1}, существование обширного набора локальных интегралов движения \autocite{Serbyn2013_2}, и специфические спектральные свойства гамильтонианов \autocite{Oganesyan2007, Serbyn2016}.
Есть также квантификаторы, характеризующие свойства собственных состояний систем с многочастичной локализацией "--- корреляции ближнего действия \autocite{Pal2010}, низкая энтропия запутанности \autocite{Bauer2013, Kjll2014, Khemani2017} и большие флуктуации локальных наблюдаемых \autocite{Bera2015}.

В контексте открытых квантовых систем явление многочастичной локализации все ещё является недостаточно хорошо изученным. Влияние диссипации на состояния систем  MBL в больших временных масштабах является очень важным направлением исследования, особенно в контексте недавних экспериментальных работ \autocite{Schreiber2015, Choi2016, Bordia2017, Smith2016}. В связанных теоретических работах \autocite{Levi2016, Fischer2016, Medvedyeva2016} рассматривались открытые системы с дефазирующей диссипацией \autocite{Poletti2013}, из-за которой состояние системы (вне зависимости от силы взаимодействия между частицами и присутствия в системе локализации) со временем приходило в тривиальное асимптотическое состояние с максимальной энтропией.

В данной диссертационной работе будет продемонстрировано, что при введении специальной физически релевантной диссипации \autocite{Diehl2008} в модель с многочастичной локализацией, система будет сходиться в новое нетривиальное асимптотическое состояние, которое может иметь следы многочастичной локализации (индуцированной свойствами многочастичного гамильтониана), даже в присутствии дефазирующей диссипации \cite{Vakulchyk2018}.  
 
Проблема взаимосвязи между классической теорией динамического хаоса и ее квантового обобщения занимает главенствующее место в круге открытых вопросов многочастичной квантовой физики, вызывая живой интерес как в области экспериментальных исследований, так и со стороны теоретиков. 

Теория диссипативного квантового хаоса активно развивается в последнее время.
В частности, в работе \autocite{Ivanchenko2017} было обобщено понятие бифуркации для случая квантовых диссипативных систем. 
В данном исследовании изучались нетривиальные асимптотические состояния периодически модулируемого открытого квантового димера с физически релевантной диссипацией \autocite{Diehl2008}.
При изменении параметра взаимодействия между квантовыми частицами в димере были обнаружены квантовые аналоги некоторых классических типов бифуркаций "--- седлоузловая, типа «вилки», а также сценарий перехода к квантовому хаосу через каскад бифуркаций удвоения периода.
 
В рамках данной диссертационной работы был впервые обнаружен квантовый аналог бифуркации Неймарка"--~Сакера \cite{Yusipov2019_1}. В своей классической вариации эта бифуркация соответствует рождению тора (инвариантная кривая в сечении Пуанкаре) из-за неустойчивости предельного цикла (неподвижной точки отображения Пуанкаре). Квантовая система демонстрирует переход от унимодального к стробоскопическому распределению в форме «бублика» для квантовых наблюдаемых.

Существует целый ряд исследований, демонстрирующих то, что неравновесные асимптотические состояния открытых квантовых систем могут иметь следы классических хаотических аттракторов в распределении Хусими, либо в структуре решения, либо в отображении Пуанкаре \autocite{Spiller1994, Brun1996, Hartmann2017, Ivanchenko2017, Carlo2017, Wang2018}. 
В то же время, количественная теория диссипативного квантового хаоса в открытых квантовых системах остаётся ещё недостаточно хорошо развитой. 
Существуют подходы состоящие в сравнительном анализе изменений спектральных свойств генераторов Линдблада или асимптотической матрицы плотности \autocite{Lindblad1976, book2007} и переходов между регулярными и хаотическими режимами \autocite{Hartmann2017, Ivanchenko2017, Grobe1988, Prosen2013}. 
Однако, данный подход имеет сложности, связанные с общностью и интерпретацией полученных результатов.

Перспективной альтернативой представляется обобщение и вычисление старшего показателя Ляпунова, положительность которого является общепризнанной характеристикой локальной неустойчивости и хаоса в классических системах. 
Однако, уравнения, описывающие открытые квантовые системы в общем случае обладают единственным устойчивым состоянием равновесия \autocite{book2007, Lucia2015} (квантовый аттрактор) и «традиционные» показатели Ляпунова в данном случае отрицательны, даже когда в системе присутствует квантовый хаос.

В данной диссертационной представлен алгоритм нахождения старшего квантового показателя Ляпунова \cite{Yusipov2019_2}, основанный на методе квантовых траекторий \autocite{Dum1992, Molmer1993, Plenio1998, Daley2014}.
Данная разработка позволяет выявить сложную структуру регулярных и хаотических областей, дать количественную оценку диссипативному квантовому хаосу. Как и классический старший показатель Ляпунова, его квантовый аналог становится положительным в случае хаотической динамики системы.

Предложенная реализация квантового показателя Ляпунова является хорошим инструментом для теоретического и численного анализа, но возникает вопрос о возможности детектировании квантового хаоса в реальном физическом эксперименте.
В данной диссертационной работе мы предлагаем экспериментально осуществимый подход к обнаружению регулярных и хаотических режимов в определённом классе открытых квантовых систем \cite{Yusipov2020}.
В модели периодически модулированного фотонного резонатора с нелинейностью Керра \autocite{Spiller1994}, при переходе в режим квантового хаоса появляется степенная асимптотика в распределении времени ожидания фотона (время между двумя последовательными испусканиями фотонов из резонатора \autocite{Vyas1989, Carmichael1989, Zhang2018}).
Это распределение может быть получено путём мониторинга эмиссии фотонов с помощью однофотонного детектора \autocite{Delteil2014, Cohen2015}, так что хаотические и регулярные состояния можно различать, не нарушая динамику внутри резонатора.

Из всего вышесказанного вытекает актуальность исследования локализации и хаоса в диссипативных квантовых системах. Помимо теоретического значения, эти результаты откроют путь для создания и управления новыми режимами в таких квантовых устройствах, как квантовые электродинамические (КЭД) резонаторы \autocite{Imamoglu1999, Walther2006, Arakawa2015} и сверхпроводящие цепи, которые являются существенно диссипативными и подвержены влиянию как внутренней, так и внешней диссипации.

{\aim} данной работы является развитие теории открытых квантовых систем, исследование таких основополагающих явлений как локализация и диссипативный квантовый хаос, разработка и реализация алгоритмов вычисления характеристик квантового диссипативного хаоса.

Для~достижения поставленной цели необходимо было решить следующие {\tasks}:
\begin{enumerate}[beginpenalty=10000] % https://tex.stackexchange.com/a/476052/104425
	\item Разработать программный комплекс для численного моделирования динамики открытых квантовых систем с большим числом состояний, включающий в себя возможность анализа отдельных квантовых траекторий, поиск асимптотических состояний системы путём численного интегрирования (при наличии модуляции в системе) или поиска собственного состояния системы, если модуляции нет.
	\item Исследовать явление локализации Андерсона в открытых квантовых системах, осуществимой при помощи физически релевантной диссипации определённого типа.
	\item Разработать методы контроля асимптотического состояния открытой квантовой системы с локализацией Андерсона, и проанализировать устойчивость полученных аттракторов к дефазирующей диссипации.
	\item Исследовать типы распространения волновых пакетов в открытых квантовых системах с локализацией Андерсона.
	\item Исследовать явление многочастичной локализации (MBL) в открытых квантовых системах.
	\item Исследовать новые типы «квантовых» бифуркаций в открытых квантовых системах.
	\item Разработать численный алгоритм поиска старшего квантового показателя Ляпунова.
	\item Исследовать экспериментально релевантные характеристики диссипативного квантового хаоса.
\end{enumerate}


{\novelty}
\begin{enumerate}[beginpenalty=10000] % https://tex.stackexchange.com/a/476052/104425
	\item Впервые были обнаружены асимптотические состояния открытых квантовых систем, несущие признаки локализации Андерсона \cite{Yusipov2017}.
	\item Продемонстрирована возможность управлять структурой квантового аттрактора с локализацией Андерсона, используя экспериментально реализуемую управляемую диссипацию. Установлена устойчивость локализации к дефазирующей диссипации \cite{Vershinina2017}.
	\item Установлена зависимость типа распространения волновых пакетов в открытых квантовых системах с локализацией Андерсона от типа управляемой диссипации \cite{Yusipov2018}.
	\item Впервые обнаружены следы многочастичной локализации в открытых квантовых системах \cite{Vakulchyk2018}.
	\item Впервые обнаружен квантовый аналог бифуркации Неймарка"--~Сакера \cite{Yusipov2019_1}.
	\item Впервые предложен и реализован алгоритм вычисления старшего квантового показателя Ляпунова на основе метода квантовых траекторий \cite{Yusipov2019_2}.
	\item Предложены новые количественные характеристики диссипативного квантового хаоса, которые могут наблюдаться в реальном физическом эксперименте \cite{Yusipov2020}.
\end{enumerate}

{\influence}. Благодаря очень быстрому прогрессу в области прикладных квантовых вычислений (все существующие на данный момент квантовые чипы, выпущенные компаниями Google, Intel, IBM, и D-Wave, состоят из сверхпроводящих кубитов), сверхпроводящие структуры оказались на переднем крае квантовой технологии. 
Сверхпроводящие чипы используются теперь не только для выполнения квантовых вычислений, но и для изучения многочастичной локализации \autocite{Roushan2017} и разнообразных фаз квантовых многочастичных систем, таким образом предоставляя новый, хорошо контролируемый и настраиваемый стенд для изучения сложных квантовых систем \autocite{Barends2015}. 
Исследование диссипативного квантового хаоса и его механизмов позволит использовать диссипативные эффекты (вместо того, чтобы бороться и подавлять их) для создания принципиально новых режимов сверхпроводящих квантовых систем, решающих задачу устойчивой обработки квантовой информации на длительных временных масштабах.

{\methods} 
Одним из самых распространённых способов описания динамики открытых квантовых систем является уравнение Линдблада \autocite{Lindblad1976, Gorini1976, book2007} для матрицы плотности.
В данной диссертационной работе будет использоваться прямое численное интегрирование (методы высоких порядков \autocite{Lambert1991}) уравнения Линдблада в случае, если в системе есть модуляция и отыскание стационарного решения как ядра диссипативного Лиувиллиана \autocite{Hartmann2017, Nation2015}. При этом для вычислений будет применяться как обычный базис состояний квантовой системы, так и специальный базис \cite{Liniov2019}, состоящий из обобщения матриц Гелл-Манна \autocite{GellMann1962} на любое количество состояний \autocite{Lendi1987} (генераторы SU(N) групп \autocite{Georgi2018}).

Также использовалось микроскопическое описание открытых квантовых систем в терминах отдельных квантовых траекторий, что позволяет «развернуть» динамику на асимптотическом квантовом состоянии "--- аттракторе "--- в ансамбль независимых реализаций (траекторий) \autocite{Dalibard1992, Dum1992, Plenio1998, Volokitin2017}

{\defpositions}
\begin{enumerate}[beginpenalty=10000] % https://tex.stackexchange.com/a/476052/104425
	\item Обнаружено явление Андерсоновской локализации в открытых квантовых системах. Диссипация может быть использована для создания нетривиальных устойчивых состояний, в которых доминируют несколько локализованных андерсоновских мод пространственно-неоднородного гамильтониана \cite{Yusipov2017}.
	\item В открытой квантовой системе с локализацией Андерсона существует механизм управления асимптотическим состоянием системы. Оно может быть локализовано в любом месте спектра гамильтониана, за счёт управляемой синтетической диссипации. Полученные таким образом состояния являются устойчивыми к дефазирующей диссипации \cite{Vershinina2017}.
	\item В открытой квантовой системе с локализацией Андерсона существуют различные типы распространения волновых пакетов. В частности, баллистический режим вызванный суммарным взаимодействием беспорядка и диссипации \cite{Yusipov2018}. 
	\item При введении специальной физически релевантной диссипации в модель с многочастичной локализацией, система будет сходиться в новое нетривиальное асимптотическое состояние, которое может иметь следы многочастичной локализации (индуцированной свойствами многочастичного гамильтониана), даже в присутствии локальной декогеренции \cite{Vakulchyk2018}.
	\item Обнаружен квантовый аналог бифуркации Неймарка"--~Сакера (рождение тора из-за неустойчивости предельного цикла) в модели открытого периодически модулируемого димера \cite{Yusipov2019_1}.
	\item Разработан и реализован алгоритм нахождения старшего квантового показателя Ляпунова, основанный на методе квантовых траекторий. Данная разработка позволяет выявить сложную структуру регулярных и хаотических областей, дать количественную оценку диссипативному квантовому хаосу. Как и классический старший показатель Ляпунова, его квантовый аналог становится положительным в случае хаотической динамики системы \cite{Yusipov2019_2}.
	\item Предложен экспериментально осуществимый подход к обнаружению регулярных и хаотических режимов в определённом классе открытых квантовых систем на основе анализа статистики распределения времени ожидания фотона. При переходе в режим квантового хаоса режиме появляется степенная асимптотика в распределении времени ожидания фотона \cite{Yusipov2020}.
\end{enumerate}

{\reliability} полученных результатов обеспечивается применением современных и принятых в научном сообществе методов численного моделирования физики открытых квантовых систем, а также сравнением результатов с работами других авторов. 
Результаты численных экспериментов полностью согласуются с теорией.

{\probation}
Основные результаты работы докладывались~на:

\begin{itemize}
	\item 22-ая Нижегородская сессия молодых учёных (естественные, математические науки) (Россия, Нижний Новгород, 23--26 мая 2017) \cite{sessiann_2017};
	\item XXI научная конференции по радиофизике (Россия, Нижний Новгород, 15--22 мая 2017) \cite{rf_2017};
	\item Третий Всероссийский молодёжный научный форум «Наука будущего – наука молодых» (Россия, Нижний Новгород, 12--15 сентября 2017) \cite{sfy_2017};
	\item Международная конференция «Shilnikov WorkShop 2017» (Россия, Нижний Новгород, 15--16 декабря 2017) \cite{shilnikov_2017};
	\item 23-я Нижегородская сессия молодых учёных (технические, естественные, математические науки) (Россия, Нижний Новгород, 22--23 мая 2017) \cite{sessiann_2018};
	\item XXII научная конференция по радиофизике, посвященная 100-летию Нижегородской радиолаборатории (Россия, Нижний Новгород, 15--29 мая 2018) \cite{rf_2018};
	\item XIII Всероссийская конференции молодых учёных «Наноэлектроника, нанофотоника и нелинейная физика» (Россия, Саратов, 4--6 сентября 2018) \cite{nnnph_2018};
	\item 9th International Scientific Conference on Physics and Control (PhysCon2019) (Россия, Иннополис, 8--11 сентября 2019) \cite{physcon_2019};
	\item International Conference «Quantization of Dissipative Chaos: Ideas and Means» (Germany, Bad-Honnef, 16--20 декабря 2019).
\end{itemize}

{\contribution} Все представленные в работе результаты были либо
получены лично автором, либо при его непосредственном участии. Автор принимал прямое участие в постановке задач, получении и анализе полученных результатов, а также в подготовке публикаций в научных журналах и представлении докладов на тематических конференциях.

\ifnumequal{\value{bibliosel}}{0}
{%%% Встроенная реализация с загрузкой файла через движок bibtex8. (При желании, внутри можно использовать обычные ссылки, наподобие `\cite{vakbib1,vakbib2}`).
    {\publications} Основные результаты по теме диссертации изложены
    в~XX~печатных изданиях,
    X из которых изданы в журналах, рекомендованных ВАК,
    X "--- в тезисах докладов.
}%
{%%% Реализация пакетом biblatex через движок biber
    \begin{refsection}[bl-author, bl-registered]
        % Это refsection=1.
        % Процитированные здесь работы:
        %  * подсчитываются, для автоматического составления фразы "Основные результаты ..."
        %  * попадают в авторскую библиографию, при usefootcite==0 и стиле `\insertbiblioauthor` или `\insertbiblioauthorgrouped`
        %  * нумеруются там в зависимости от порядка команд `\printbibliography` в этом разделе.
        %  * при использовании `\insertbiblioauthorgrouped`, порядок команд `\printbibliography` в нём должен быть тем же (см. biblio/biblatex.tex)
        %
        % Невидимый библиографический список для подсчёта количества публикаций:
        \printbibliography[heading=nobibheading, section=1, env=countauthorvak,          keyword=biblioauthorvak]%
        \printbibliography[heading=nobibheading, section=1, env=countauthorwos,          keyword=biblioauthorwos]%
        \printbibliography[heading=nobibheading, section=1, env=countauthorscopus,       keyword=biblioauthorscopus]%
        \printbibliography[heading=nobibheading, section=1, env=countauthorconf,         keyword=biblioauthorconf]%
        \printbibliography[heading=nobibheading, section=1, env=countauthorother,        keyword=biblioauthorother]%
        \printbibliography[heading=nobibheading, section=1, env=countregistered,         keyword=biblioregistered]%
        \printbibliography[heading=nobibheading, section=1, env=countauthorpatent,       keyword=biblioauthorpatent]%
        \printbibliography[heading=nobibheading, section=1, env=countauthorprogram,      keyword=biblioauthorprogram]%
        \printbibliography[heading=nobibheading, section=1, env=countauthor,             keyword=biblioauthor]%
        \printbibliography[heading=nobibheading, section=1, env=countauthorvakscopuswos, filter=vakscopuswos]%
        \printbibliography[heading=nobibheading, section=1, env=countauthorscopuswos,    filter=scopuswos]%
        %
        \nocite{*}%
        %
        {\publications} Основные результаты по теме диссертации изложены в~\arabic{citeauthor}~печатных изданиях,
        \arabic{citeauthorvak} из которых изданы в журналах, рекомендованных ВАК\sloppy%
        \ifnum \value{citeauthorscopuswos}>0%
            , все из которых входят в списки Web of~Science и Scopus\sloppy%
        \fi%
        \ifnum \value{citeauthorconf}>0%
            , \arabic{citeauthorconf} "--- в~тезисах докладов.
        \else%
            .
        \fi%
        \ifnum \value{citeregistered}=1%
            \ifnum \value{citeauthorpatent}=1%
                %Зарегистрирован \arabic{citeauthorpatent} патент.
            \fi%
            \ifnum \value{citeauthorprogram}=1%
                Зарегистрирована \arabic{citeauthorprogram} программа для ЭВМ.
            \fi%
        \fi%
        \ifnum \value{citeregistered}>1%
            Зарегистрированы\ %
            \ifnum \value{citeauthorpatent}>0%
            \formbytotal{citeauthorpatent}{патент}{}{а}{}\sloppy%
            \ifnum \value{citeauthorprogram}=0 . \else \ и~\fi%
            \fi%
            \ifnum \value{citeauthorprogram}>0%
            \formbytotal{citeauthorprogram}{программ}{а}{ы}{} для ЭВМ.
            \fi%
        \fi%
        % К публикациям, в которых излагаются основные научные результаты диссертации на соискание учёной
        % степени, в рецензируемых изданиях приравниваются патенты на изобретения, патенты (свидетельства) на
        % полезную модель, патенты на промышленный образец, патенты на селекционные достижения, свидетельства
        % на программу для электронных вычислительных машин, базу данных, топологию интегральных микросхем,
        % зарегистрированные в установленном порядке.(в ред. Постановления Правительства РФ от 21.04.2016 N 335)
    \end{refsection}%
    \begin{refsection}[bl-author, bl-registered]
        % Это refsection=2.
        % Процитированные здесь работы:
        %  * попадают в авторскую библиографию, при usefootcite==0 и стиле `\insertbiblioauthorimportant`.
        %  * ни на что не влияют в противном случае
        %\nocite{vakbib2}%vak
        %\nocite{patbib1}%patent
        %\nocite{progbib1}%program
        %\nocite{bib1}%other
        %\nocite{confbib1}%conf
    \end{refsection}%
        %
        % Всё, что вне этих двух refsection, это refsection=0,
        %  * для диссертации - это нормальные ссылки, попадающие в обычную библиографию
        %  * для автореферата:
        %     * при usefootcite==0, ссылка корректно сработает только для источника из `external.bib`. Для своих работ --- напечатает "[0]" (и даже Warning не вылезет).
        %     * при usefootcite==1, ссылка сработает нормально. В авторской библиографии будут только процитированные в refsection=0 работы.
}