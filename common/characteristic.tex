
{\actuality} 
В настоящее время диссипативные квантовые системы являются неотъемлемой частью экспериментальной и технологической реальности.
Практические реализации нано- \autocite{Poot2012} и оптомеханических систем \autocite{Aspelmeyer2014}, сверхпроводящих элементов \autocite{Clarke2008} не осуществимы в полной изоляции от окружающей среды.
Эволюция идеальных когерентных квантовых систем является простым приближением реальных физических процессов. 
Все эти системы являются открытыми и их динамика существенно диссипативна.
Диссипация в данных системах является полноправным генератором эволюции, которая не менее сложна и разнообразна, чем унитарная, генерируемая гамильтонианами.
Асимптотические состояния открытых квантовых систем определится не только гамильтоновой динамикой, но взаимодействием с окружающей средой.
Данное взаимодействие не является достаточно сильным, чтобы полностью свести динамику системы к классическому случаю \autocite{Breuer2007}.
Особенностью диссипации является то, что она способна привести систему в состояния, которые недостижимы в классическом пределе.
Например, новые топологические состояния, получаемые за счёт управляемой синтетической диссипации \autocite{Diehl2011} или «чистые» сильно запутанные состояния в многочастичных квантовых системах \autocite{Kraus2008}.

На сегодняшний день изучение открытых квантовых систем является активно развивающейся областью теоретической и экспериментальной физики с широким спектром разноплановых фундаментальных результатов и методов \autocite{Breuer2007}.
Эксперименты с открытыми квантовыми системами являются трудоёмкими и могут быть выполнены только в специальных, весьма ограниченных условиях.
Поэтому основным подходом теоретического изучения физики открытых квантовых систем является математическое моделирование.
В частности, одним из самых распространённых подходов к описанию динамики открытых квантовых систем является формализм Линдблада \autocite{Lindblad1976, Gorini1976}.
Поиск аналитических решений существующих математических моделей осложнён чрезвычайно высокой размерностью получаемых систем дифференциальных уравнений.
Из-за этого основным инструментом изучения открытых квантовых систем является численное моделирование.
Актуальным также является разработка новых численных характеристик и методов анализа.
Среди активно развиваемых и наиболее удачных подходов с точки зрения параллелизма вычислений является метод квантовых скачков \autocite{Dalibard1992, Dum1992, Plenio1998}.
Помимо вычислительных преимуществ, данный подход является популярным инструментом в сфере квантовой оптики \autocite{Carmichael1993}, имеет приложения в вычислительной физике холодных атомов \autocite{Diehl2011, Diehl2008, Marcuzzi2014} и активно используется в контексте квантовых электродинамических (КЭД) систем \autocite{Imamoglu1999, Walther2006, Arakawa2015}.

В теории открытых квантовых систем существует ряд фундаментальных проблем, которые ещё не получили должного математического описания.
В центре внимания данной работы находятся две из них: локализация и хаос.
Данные проблемы являются актуальными и активно обсуждаются на различных конференциях мирового уровня, в частности «Dissipative Quantum Chaos: from Semi-Groups to QED Experiments» (Daejeon, South Korea, октябрь 2017) и «Quantization of Dissipative Chaos: Ideas and Means» (Bad Honnef, Germany, декабрь 2019).

Теория одночастичной и многочастичной локализации является хорошо развитой для изолированных квантовых систем, а также нелинейных классических систем, получаемых из квантовых в приближении среднего поля.
Есть указания на то, что свойства локализации могут проявляться в открытых квантовых системах, но все что было обнаружено в предшествующих исследованиях касается переходных процессов\autocite{Genway2014}.
Возникает целый ряд вопросов о существовании свойств одночастичной и многочастичной локализации в асимптотических состояниях открытых квантовых систем, о специфике математических моделей, в которых это может проявляться, о численных методах, позволяющих это обнаружить. 
Ответы на данные вопросы до недавнего времени оставалось неизвестными.

Теория диссипативного квантового хаоса является активно развивающейся в последнее время.
В частности, в работе \autocite{Hartmann2017, Ivanchenko2017} было обобщено понятие бифуркации для случая квантовых диссипативных систем и обнаружены некоторые аналоги классических бифуркаций.
Однако, возникают вопросы о существовании других классических бифуркационных сценариев в открытых квантовых системах, о математических моделях в которых они могут быть обнаружены, и методах их обнаружения.
Ещё одним важным направлением исследований является создание математических методов (и их программная реализация) для количественной оценки диссипативного квантового хаоса.
Отдельный интерес представляет исследование математических моделей, описывающих физические системы, в которых количественные характеристики квантового хаоса можно получить из данных реального эксперимента.

Из всего вышесказанного вытекает актуальность исследования локализации и хаоса в диссипативных квантовых системах. Помимо теоретического значения, эти результаты откроют путь для создания и управления новыми режимами в таких квантовых устройствах, как квантовые электродинамические (КЭД) резонаторы \autocite{Imamoglu1999, Walther2006, Arakawa2015} и сверхпроводящие цепи, которые являются существенно диссипативными.

{\aim} данной работы является развитие математических моделей, численных методов и их программных реализаций, позволяющих получить, исследовать и охарактеризовать явления локализации и хаоса в открытых квантовых системах.

Для~достижения поставленной цели необходимо было решить следующие {\tasks}:
\begin{enumerate}[beginpenalty=10000] % https://tex.stackexchange.com/a/476052/104425
	\item Разработать программный комплекс для численного моделирования динамики открытых квантовых систем с большим числом состояний, включающий в себя возможность анализа отдельных квантовых траекторий, поиск асимптотических состояний системы путём численного интегрирования (при наличии модуляции в системе) или поиска собственного состояния системы, если модуляции нет.
	\item Идентифицировать и численно исследовать математические модели открытых квантовых систем, которые допускают существование одночастичной локализации в асимптотических состояниях. Разработать методы управления локализационными свойствами таких систем. Численно моделировать распространения волновых пакетов в системах с одночастичной локализацией, выявить особенности данных процессов в зависимости от степени локализации.
	\item Численно моделировать открытые квантовые системы с многочастичной локализацией в асимптотических состояниях, исследовать количественные признаки данного явления.
	\item Идентифицировать и численно исследовать математические модели открытых квантовых систем, в которых существуют новые квантовые аналоги классических бифуркаций.
	\item Разработать численный алгоритм поиска старшего квантового показателя Ляпунова, позволяющего количественно оценить динамику открытой квантовой системы.
	\item Разработать квантификаторы диссипативного квантового хаоса, которые могут быть доступны в реальном эксперименте. Численно исследовать математические модели соответствующих открытых квантовых систем. Изучить способы контролирования динамики открытых квантовых систем на основе спин-фотонного взаимодействия. 
\end{enumerate}


{\novelty}
\begin{enumerate}[beginpenalty=10000] % https://tex.stackexchange.com/a/476052/104425
	\item Впервые предложена и численно исследована математическая модель открытой квантовой системы с признаками одночастичной локализации в асимптотических состояниях \cite{Yusipov2017}. 
	Предложен метод управления структурой квантового аттрактора с признаками локализации, используя экспериментально реализуемую, управляемую диссипацию. 
	Установлена устойчивость локализации к дефазирующей диссипации \cite{Vershinina2017}.
	Численно исследована зависимость типа распространения волновых пакетов в открытых квантовых системах с локализацией от типа управляемой диссипации \cite{Yusipov2018}.
	\item Произведён численный анализ открытых квантовых систем с признаками многочастичной локазации. Предложены численные характеристики данного явления \cite{Vakulchyk2018}.
	\item Численно исследована модель открытого квантового димера, в который впервые был обнаружен квантовый аналог классической бифуркации Неймарка"--~Сакера \cite{Yusipov2019_1} (рождение инвариантного тора из-за неустойчивости предельного цикла).
	\item Впервые предложен и реализован алгоритм вычисления старшего квантового показателя Ляпунова на основе метода квантовых траекторий \cite{Yusipov2019_2}.
	\item Предложены новые количественные характеристики диссипативного квантового хаоса, которые могут наблюдаться в реальном физическом эксперименте \cite{Yusipov2020}. Численно исследована модель открытого квантового резонатора со спином, в которой существует возможность контролировать тип динамики системы \cite{Yusipov2021}.
\end{enumerate}

{\influence}. Благодаря очень быстрому прогрессу в области прикладных квантовых вычислений (все существующие на данный момент квантовые чипы, выпущенные компаниями Google, Intel, IBM, и D-Wave, состоят из сверхпроводящих кубитов), сверхпроводящие структуры оказались на переднем крае квантовой технологии. 
Сверхпроводящие чипы используются теперь не только для выполнения квантовых вычислений, но и для изучения многочастичной локализации \autocite{Roushan2017} и разнообразных фаз квантовых многочастичных систем, таким образом предоставляя новый, хорошо контролируемый и настраиваемый стенд для изучения сложных квантовых систем \autocite{Barends2015}. 
Исследование диссипативного квантового хаоса и его механизмов позволит использовать диссипативные эффекты (вместо того, чтобы бороться и подавлять их) для создания принципиально новых режимов сверхпроводящих квантовых систем, решающих задачу устойчивой обработки квантовой информации на длительных временных масштабах.

{\methods} 
Математические модели открытых квантовых систем были получены при в рамках формализма Линдблада \autocite{Lindblad1976, Gorini1976, book2007} для матрицы плотности.
Численный анализ полученных моделей осуществлялся посредством прямого численное интегрирования (методы высоких порядков \autocite{Lambert1991}) уравнения Линдблада в случае, если в системе есть модуляция или посредством численного поиска собственного вектора, соответствующего нулевому собственному числу, матрицы оператора Линдблада (если модуляции нет). При этом для вычислений использовался как классический базис гильбертова пространства состояний квантовой системы, так и специальный базис \cite{Liniov2019}, состоящий из обобщения матриц Гелл-Манна \autocite{GellMann1962} на любое количество состояний \autocite{Lendi1987} (генераторы SU(N) групп \autocite{Georgi2018}).
Также применялось микроскопическое описание открытых квантовых систем в терминах отдельных квантовых траекторий, что позволяло  «развернуть» динамику на асимптотическом квантовом состоянии "--- аттракторе "--- в ансамбль независимых реализаций (траекторий) \autocite{Dalibard1992, Dum1992, Plenio1998, Volokitin2017}.

{\defpositions}
\begin{enumerate}[beginpenalty=10000] % https://tex.stackexchange.com/a/476052/104425
	\item Реализован программный комплекс на языке C++, осуществляющий численное моделирование открытых квантовых систем. Асимптотические состояния находятся путём численного интегрирования уравнения Линдблада или путём решения задачи по поиску собственных векторов и значений для матрицы оператора Линдблада. Программный комплекс предусматривает возможность анализа отдельных квантовых траекторий.
	\item Найдена и численно исследована математическая модель открытой квантовой системы с признаками одночастичной локализации в асимптотических состояниях \cite{Yusipov2017}. Реализован метод управления локализационными свойствами получаемого квантового аттрактора \cite{Vershinina2017}. Изучены характеристики волновых пакетов отдельных квантовых траекторий в асимптотическом режиме \cite{Yusipov2018}.
	\item Численно исследована математическая модель открытой квантовой системы с многочастичной локализацией. Предложены количественные критерии данного явления \cite{Vakulchyk2018}.
	\item В ходе численного моделирования открытого периодически модулируемого квантового димера обнаружен квантовый аналог бифуркации Неймарка"--~Сакера (рождение тора из-за неустойчивости предельного цикла) \cite{Yusipov2019_1}.
	\item Разработан и программно реализован \cite{prog1} алгоритм нахождения старшего квантового показателя Ляпунова, основанный на методе квантовых траекторий. Данная разработка позволяет выявить сложную структуру регулярных и хаотических областей, дать количественную оценку диссипативному квантовому хаосу. Как и классический старший показатель Ляпунова, его квантовый аналог становится положительным в случае хаотической динамики системы \cite{Yusipov2019_2}.
	\item Предложен экспериментально осуществимый подход к обнаружению регулярных и хаотических режимов в определённом классе математических моделей открытых квантовых систем на основе анализа статистики распределения времени ожидания фотона. При переходе в режим квантового хаоса режиме появляется степенная асимптотика в распределении времени ожидания фотона \cite{Yusipov2020}. Также численно исследованы математические модели со спин-фотонным взаимодействием, позволяющим регулировать динамику в системе \cite{Yusipov2021}.
\end{enumerate}

{\reliability} полученных результатов обеспечивается применением современных и принятых в научном сообществе методов численного моделирования физики открытых квантовых систем, а также сравнением результатов с работами других авторов. 
Результаты численных экспериментов полностью согласуются с теорией.

{\probation}
Основные результаты работы докладывались~на:

\begin{itemize}
	\item 22-ая Нижегородская сессия молодых учёных (естественные, математические науки) (Россия, Нижний Новгород, 23--26 мая 2017) \cite{sessiann_2017};
	\item XXI научная конференции по радиофизике (Россия, Нижний Новгород, 15--22 мая 2017) \cite{rf_2017};
	\item Третий Всероссийский молодёжный научный форум «Наука будущего – наука молодых» (Россия, Нижний Новгород, 12--15 сентября 2017) \cite{sfy_2017};
	\item Международная конференция «Shilnikov WorkShop 2017» (Россия, Нижний Новгород, 15--16 декабря 2017) \cite{shilnikov_2017};
	\item 23-я Нижегородская сессия молодых учёных (технические, естественные, математические науки) (Россия, Нижний Новгород, 22--23 мая 2017) \cite{sessiann_2018};
	\item XXII научная конференция по радиофизике, посвященная 100-летию Нижегородской радиолаборатории (Россия, Нижний Новгород, 15--29 мая 2018) \cite{rf_2018};
	\item XIII Всероссийская конференции молодых учёных «Наноэлектроника, нанофотоника и нелинейная физика» (Россия, Саратов, 4--6 сентября 2018) \cite{nnnph_2018};
	\item 9th International Scientific Conference on Physics and Control (PhysCon2019) (Россия, Иннополис, 8--11 сентября 2019) \cite{physcon_2019};
	\item International Conference «Quantization of Dissipative Chaos: Ideas and Means» (Germany, Bad-Honnef, 16--20 декабря 2019).
	\item XXIV научная конференция по радиофизике, посвященная 75-летию радиофизического факультета (Россия, Нижний Новгород, 13--31 мая 2020) \cite{rf_2020};
\end{itemize}

{\contribution} Все представленные в работе результаты были либо
получены лично автором, либо при его непосредственном участии. Автор принимал прямое участие в постановке задач, получении и анализе полученных результатов, а также в подготовке публикаций в научных журналах и представлении докладов на тематических конференциях.

\ifnumequal{\value{bibliosel}}{0}
{%%% Встроенная реализация с загрузкой файла через движок bibtex8. (При желании, внутри можно использовать обычные ссылки, наподобие `\cite{vakbib1,vakbib2}`).
    {\publications} Основные результаты по теме диссертации изложены
    в~XX~печатных изданиях,
    X из которых изданы в журналах, рекомендованных ВАК,
    X "--- в тезисах докладов.
}%
{%%% Реализация пакетом biblatex через движок biber
    \begin{refsection}[bl-author, bl-registered]
        % Это refsection=1.
        % Процитированные здесь работы:
        %  * подсчитываются, для автоматического составления фразы "Основные результаты ..."
        %  * попадают в авторскую библиографию, при usefootcite==0 и стиле `\insertbiblioauthor` или `\insertbiblioauthorgrouped`
        %  * нумеруются там в зависимости от порядка команд `\printbibliography` в этом разделе.
        %  * при использовании `\insertbiblioauthorgrouped`, порядок команд `\printbibliography` в нём должен быть тем же (см. biblio/biblatex.tex)
        %
        % Невидимый библиографический список для подсчёта количества публикаций:
        \printbibliography[heading=nobibheading, section=1, env=countauthorvak,          keyword=biblioauthorvak]%
        \printbibliography[heading=nobibheading, section=1, env=countauthorwos,          keyword=biblioauthorwos]%
        \printbibliography[heading=nobibheading, section=1, env=countauthorscopus,       keyword=biblioauthorscopus]%
        \printbibliography[heading=nobibheading, section=1, env=countauthorconf,         keyword=biblioauthorconf]%
        \printbibliography[heading=nobibheading, section=1, env=countauthorother,        keyword=biblioauthorother]%
        \printbibliography[heading=nobibheading, section=1, env=countregistered,         keyword=biblioregistered]%
        \printbibliography[heading=nobibheading, section=1, env=countauthorpatent,       keyword=biblioauthorpatent]%
        \printbibliography[heading=nobibheading, section=1, env=countauthorprogram,      keyword=biblioauthorprogram]%
        \printbibliography[heading=nobibheading, section=1, env=countauthor,             keyword=biblioauthor]%
        \printbibliography[heading=nobibheading, section=1, env=countauthorvakscopuswos, filter=vakscopuswos]%
        \printbibliography[heading=nobibheading, section=1, env=countauthorscopuswos,    filter=scopuswos]%
        %
        \nocite{*}%
        %
        {\publications} Основные результаты по теме диссертации изложены в~\arabic{citeauthor}~печатных изданиях\sloppy%
        \ifnum \value{citeauthorscopuswos}>0%
        	, \arabic{citeauthorscopuswos} "--- в~периодических научных журналах, индексируемых Web of~Science и Scopus\sloppy%
        \fi%
        \ifnum \value{citeauthorconf}>0%
            , \arabic{citeauthorconf} "--- в~тезисах докладов.
        \else%
            .
        \fi%
        \ifnum \value{citeregistered}=1%
            \ifnum \value{citeauthorpatent}=1%
                Зарегистрирован \arabic{citeauthorpatent} патент.
            \fi%
            \ifnum \value{citeauthorprogram}=1%
                Зарегистрирована \arabic{citeauthorprogram} программа для ЭВМ.
            \fi%
        \fi%
        \ifnum \value{citeregistered}>1%
            Зарегистрированы\ %
            \ifnum \value{citeauthorpatent}>0%
            \formbytotal{citeauthorpatent}{патент}{}{а}{}\sloppy%
            \ifnum \value{citeauthorprogram}=0 . \else \ и~\fi%
            \fi%
            \ifnum \value{citeauthorprogram}>0%
            \formbytotal{citeauthorprogram}{программ}{а}{ы}{} для ЭВМ.
            \fi%
        \fi%
        % К публикациям, в которых излагаются основные научные результаты диссертации на соискание учёной
        % степени, в рецензируемых изданиях приравниваются патенты на изобретения, патенты (свидетельства) на
        % полезную модель, патенты на промышленный образец, патенты на селекционные достижения, свидетельства
        % на программу для электронных вычислительных машин, базу данных, топологию интегральных микросхем,
        % зарегистрированные в установленном порядке.(в ред. Постановления Правительства РФ от 21.04.2016 N 335)
    \end{refsection}%
    \begin{refsection}[bl-author, bl-registered]
        % Это refsection=2.
        % Процитированные здесь работы:
        %  * попадают в авторскую библиографию, при usefootcite==0 и стиле `\insertbiblioauthorimportant`.
        %  * ни на что не влияют в противном случае
        %\nocite{vakbib2}%vak
        %\nocite{patbib1}%patent
        %\nocite{progbib1}%program
        %\nocite{bib1}%other
        %\nocite{confbib1}%conf
    \end{refsection}%
        %
        % Всё, что вне этих двух refsection, это refsection=0,
        %  * для диссертации - это нормальные ссылки, попадающие в обычную библиографию
        %  * для автореферата:
        %     * при usefootcite==0, ссылка корректно сработает только для источника из `external.bib`. Для своих работ --- напечатает "[0]" (и даже Warning не вылезет).
        %     * при usefootcite==1, ссылка сработает нормально. В авторской библиографии будут только процитированные в refsection=0 работы.
}