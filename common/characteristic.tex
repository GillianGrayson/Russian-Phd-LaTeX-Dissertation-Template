
{\actuality} 
В настоящее время диссипативные квантовые системы являются неотъемлемой частью экспериментальной и технологической реальности.
Практические реализации нано- \autocite{Poot2012} и опто-механических систем \autocite{Aspelmeyer2014}, сверхпроводящих элементов \autocite{Clarke2008} не осуществимы в полной изоляции от окружающей среды.
Эволюция идеальных когерентных квантовых систем является простым приближением реальных физических процессов. 
Все эти системы являются открытыми и их динамика существенно диссипативна.
Диссипация в данных системах является полноправным генератором эволюции, которая не менее сложна и разнообразна, чем унитарная, генерируемая гамильтонианами.
Асимптотические состояния открытых квантовых систем определится не только гамильтоновой динамикой, но взаимодействием с окружающей средой.
Данное взаимодействие не является достаточно сильным, чтобы полностью свести динамику системы к классическому случаю \autocite{Breuer2007}.
Особенностью диссипации является то, что она способна привести систему в состояния, которые недостижимы в классическом пределе.
Например, новые топологические состояния, получаемые за счёт управляемой синтетической диссипации \autocite{Diehl2011} или «чистые» сильно запутанные состояния в многочастичных квантовых системах \autocite{Kraus2008}.

На сегодняшний день теория открытых квантовых систем является хорошо развитой областью теоретической физики с широким спектром разноплановых фундаментальных результатов и методов \autocite{Breuer2007}.
Одним из самых распространённых подходов к описанию динамики открытых квантовых систем является формализм Линдблада \autocite{Lindblad1976, Gorini1976} и метод квантовых скачков \autocite{Dalibard1992, Dum1992, Plenio1998}, являющийся популярным инструментом в сфере квантовой оптики \autocite{Carmichael1993}.
Кроме того, этот метод имеет приложения в физике холодных атомов \autocite{Diehl2011, Diehl2008, Marcuzzi2014} и активно используется в контексте квантовых электродинамических (КЭД) систем.

Асимптотическое состояние открытых квантовых систем полностью определяется типом взаимодействия с окружающей средой диссипации. 
Например, диссипация описанная в работе \autocite{Poletti2013} приводит любую квантовую систему в тривиальное состояние с максимальной энтропией (все состояния системы становятся равновероятными).
В настоящее время доминирующей тенденцией является идея «борьбы» с декогеренцией (например, увеличение времени когерентности атомных конденсатов в оптических решётках). Однако, все большее распространение получает идея диссипативной инженерии \autocite{Diehl2008}, позволяющей создавать принципиально новые состояния многочастичных систем.
В частности, в работе \autocite{Ivanchenko2017} исследуются асимптотические состояния периодически модулируемого открытого квантового димера с данным типом диссипации.
При изменении параметра взаимодействия между квантовыми частицами были обнаружены квантовые аналоги некоторых классических типов бифуркаций "--- седлоузловая, типа «вилки», а также сценарий перехода к квантовому хаосу через каскад бифуркаций удвоения периода.

Однако, существуют области физики открытых квантовых систем, которые не являются достаточно полно изученными. В частности, явление локализации Андерсона в пространственно-неоднородных средах \autocite{Anderson1958, Kramer1993, Evers2008, Esposito2012}, которая является очень хорошо изученным феноменом в случае невзаимодействующих частиц в когерентном гамильтоновом пределе \autocite{Segev2013, Billy2008, Roati2008, Yedjour2010, Kondov2011, Jendrzejewski2012} и гораздо менее изучена в случае взаимодействий системы с окружающей средой \autocite{Breuer2007}. Другие типы локализации, индуцированные топологией плоских зон \autocite{Derzhko2006, Bergman2008, Flach2014}, также вызваны деструктивной интерференцией, препятствующей распространению волновых пакетов. Следовательно, можно ожидать, что данные типы локализации также будут исчезать при наличии в системе диссипации.

Предположение о том, что в открытых системах диссипация должна оказывать деструктивное влияние на интерференцию и разрушать локализацию Андерсона \autocite{Genway2014}, не всегда выполнимо.
Первые доказательства того, что локализация Андерсона может существовать в диссипативных системах в асимптотическом пределе, были получены для квазиклассических и классических систем. В частности, были получены экспериментальные свидетельства локализации в случайном лазере, за счёт которой уменьшается пространственное перекрытие мод и, в результате, улучшается стабильность лазера \autocite{Stano2012, Liu2014}.

В рамках данной диссертационной работы будет продемонстрирована возможность существования асимптотических состояний в открытой квантовой системе, которые сохраняют признаки локализации Андерсона \cite{Yusipov2017}. Такое асимптотическое состояние может быть спроектировано с помощью некоторого набора параметризованных диссипативных операторов (диссипаторов) \autocite{Diehl2008}, осуществляющих отбор локализованных в пространстве андерсоновских мод.
Изменяя параметры данных диссипаторов можно контролировать структуру аттракторов путём избирательной селекции собственных состояний гамильтониана \cite{Vershinina2017}. В результате в неупорядоченной системе Андерсона асимптотическое состояние может быть локализовано в любом месте спектра гамильтониана. При этом локализация будет сохраняться в случае наличия в системе добавочной дефазирующей диссипации.

Интересным является вопрос о типе распространения волновых пакетов в рассматриваемых открытых квантовых системах с локализацией Андерсона.
В основополагающей работе о локализации за счёт беспорядка \autocite{Anderson1958} было отмечено, что взаимодействие системы с локализующим гамильтонианом с окружающей средой должно оказывать сильное влияние на результирующую динамику. 
В ранее проведённых исследованиях \autocite{Gurvitz2000, Nowak2012, Flores1999} было установлено, что декогеренция уничтожает все следы локализации и приводит систему к тривиальному режиму нормальной диффузии.
Недавние результаты, однако, демонстрируют гораздо более богатую физику: даже когда асимптотическое состояние является тривиальным равномерным распределением, процесс релаксации проявляет неоднородную динамику с признаками метастабильности \autocite{Genway2014}. А в работах \autocite{Valenti2015, Spagnolo2015, Spagnolo2016} есть указания на то, что диссипация может даже стабилизировать метастабильные состояния.

Представленные в данной диссертационной работе результаты продемонстрируют, что диффузия волновых пакетов имеет место лишь для ограниченного класса диссипаторов, которые в самом деле уничтожают все признаки локализации в асимптотическом состоянии.
В случае использования других параметров диссипативных операторов, будут обнаружены нетривиальные режимы баллистического распространения волновых пакетов, механизм возникновения которых связан с конкуренцией между
диссипацией и беспорядком \cite{Yusipov2018}.

Помимо классической локализации Андерсона \autocite{Anderson1958}, можно рассмотреть ее расширение в контексте систем с большим количеством частиц "--- многочастичную локализацию (MBL) \autocite{Gornyi2005, Basko2006}.
Существует целый спектр определений и квантификаторов этого многогранного явления, нацеленных на выделение специфических свойств систем с многочастичной локализацией.
Среди них можно выделить отсутствие проводимости \autocite{Gornyi2005} (даже в пределе бесконечной температуры \autocite{Basko2006}), медленный логарифмический рост энтропии запутанности при уменьшении параметра взаимодействия между частицами \autocite{Chiara2006, Znidaric2008, Bardarson2012, Serbyn2013_1}, существование обширного набора локальных интегралов движения \autocite{Serbyn2013_2}, и специфические спектральные свойства гамильтонианов \autocite{Oganesyan2007, Serbyn2016}.
Есть также квантификаторы, характеризующие свойства собственных состояний систем с многочастичной локализацией "--- корреляции ближнего действия \autocite{Pal2010}, низкая энтропия запутывания \autocite{Bauer2013, Kjll2014, Khemani2017} и большие флуктуации локальных наблюдаемых \autocite{Bera2015}.

В контексте открытых квантовых систем явление многочастичной локализации все ещё является недостаточно изученным. Влияние диссипации на состояния систем  MBL в больших временных масштабах является очень важным направлением исследования, особенно в контексте недавних экспериментальных работ \autocite{Schreiber2015, Choi2016, Bordia2017, Smith2016}. В связанных теоретических работах \autocite{Levi2016, Fischer2016, Medvedyeva2016} рассматривались открытые системы с дефазирующей диссипацией \autocite{Poletti2013}, из-за которой состояние системы (вне зависимости от силы взаимодействия между частицами и присутствия в системе локализации) со временем приходило в тривиальное асимптотическое состояние с максимальной энтропией.

В данной диссертационной работе будет продемонстрировано, что при введении специальной физически релевантной диссипации \autocite{Diehl2008} в модель с многочастичной локализацией, система будет сходиться в новое нетривиальное асимптотическое состояние, которое может иметь следы многочастичной локализации (индуцированной свойствами многочастичного гамильтониана), даже в присутствии локальной декогеренции \cite{Vakulchyk2018}.  
 
 
 
 
 
 
 
 
 
 
Обзор, введение в тему, обозначение места данной работы в
мировых исследованиях и~т.\:п., можно использовать ссылки на~другие
работы~\autocite{Gosele1999161,Lermontov}
(если их~нет, то~в~автореферате
автоматически пропадёт раздел <<Список литературы>>). Внимание! Ссылки
на~другие работы в~разделе общей характеристики работы можно
использовать только при использовании \verb!biblatex! (из-за технических
ограничений \verb!bibtex8!. Это связано с тем, что одна
и~та~же~характеристика используются и~в~тексте диссертации, и в
автореферате. В~последнем, согласно ГОСТ, должен присутствовать список
работ автора по~теме диссертации, а~\verb!bibtex8! не~умеет выводить в~одном
файле два списка литературы).
При использовании \verb!biblatex! возможно использование исключительно
в~автореферате подстрочных ссылок
для других работ командой \verb!\autocite!, а~также цитирование
собственных работ командой \verb!\cite!. Для этого в~файле
\verb!common/setup.tex! необходимо присвоить положительное значение
счётчику \verb!\setcounter{usefootcite}{1}!.

Для генерации содержимого титульного листа автореферата, диссертации
и~презентации используются данные из файла \verb!common/data.tex!. Если,
например, вы меняете название диссертации, то оно автоматически
появится в~итоговых файлах после очередного запуска \LaTeX. Согласно
ГОСТ 7.0.11-2011 <<5.1.1 Титульный лист является первой страницей
диссертации, служит источником информации, необходимой для обработки и
поиска документа>>. Наличие логотипа организации на~титульном листе
упрощает обработку и~поиск, для этого разметите логотип вашей
организации в папке images в~формате PDF (лучше найти его в векторном
варианте, чтобы он хорошо смотрелся при печати) под именем
\verb!logo.pdf!. Настроить размер изображения с логотипом можно
в~соответствующих местах файлов \verb!title.tex!  отдельно для
диссертации и автореферата. Если вам логотип не~нужен, то просто
удалите файл с~логотипом.

\ifsynopsis
Этот абзац появляется только в~автореферате.
Для формирования блоков, которые будут обрабатываться только в~автореферате,
заведена проверка условия \verb!\!\verb!ifsynopsis!.
Значение условия задаётся в~основном файле документа (\verb!synopsis.tex! для
автореферата).
\else
Этот абзац появляется только в~диссертации.
Через проверку условия \verb!\!\verb!ifsynopsis!, задаваемого в~основном файле
документа (\verb!dissertation.tex! для диссертации), можно сделать новую
команду, обеспечивающую появление цитаты в~диссертации, но~не~в~автореферате.
\fi

% {\progress}
% Этот раздел должен быть отдельным структурным элементом по
% ГОСТ, но он, как правило, включается в описание актуальности
% темы. Нужен он отдельным структурынм элемементом или нет ---
% смотрите другие диссертации вашего совета, скорее всего не нужен.

{\aim} данной работы является \ldots

Для~достижения поставленной цели необходимо было решить следующие {\tasks}:
\begin{enumerate}[beginpenalty=10000] % https://tex.stackexchange.com/a/476052/104425
  \item Исследовать, разработать, вычислить и~т.\:д. и~т.\:п.
  \item Исследовать, разработать, вычислить и~т.\:д. и~т.\:п.
  \item Исследовать, разработать, вычислить и~т.\:д. и~т.\:п.
  \item Исследовать, разработать, вычислить и~т.\:д. и~т.\:п.
\end{enumerate}


{\novelty}
\begin{enumerate}[beginpenalty=10000] % https://tex.stackexchange.com/a/476052/104425
  \item Впервые \ldots
  \item Впервые \ldots
  \item Было выполнено оригинальное исследование \ldots
\end{enumerate}

{\influence} \ldots

{\methods} \ldots

{\defpositions}
\begin{enumerate}[beginpenalty=10000] % https://tex.stackexchange.com/a/476052/104425
  \item Первое положение
  \item Второе положение
  \item Третье положение
  \item Четвертое положение
\end{enumerate}
В папке Documents можно ознакомиться в решением совета из Томского ГУ
в~файле \verb+Def_positions.pdf+, где обоснованно даются рекомендации
по~формулировкам защищаемых положений.

{\reliability} полученных результатов обеспечивается \ldots \ Результаты находятся в соответствии с результатами, полученными другими авторами.


{\probation}
Основные результаты работы докладывались~на:
перечисление основных конференций, симпозиумов и~т.\:п.

{\contribution} Автор принимал активное участие \ldots

\ifnumequal{\value{bibliosel}}{0}
{%%% Встроенная реализация с загрузкой файла через движок bibtex8. (При желании, внутри можно использовать обычные ссылки, наподобие `\cite{vakbib1,vakbib2}`).
    {\publications} Основные результаты по теме диссертации изложены
    в~XX~печатных изданиях,
    X из которых изданы в журналах, рекомендованных ВАК,
    X "--- в тезисах докладов.
}%
{%%% Реализация пакетом biblatex через движок biber
    \begin{refsection}[bl-author, bl-registered]
        % Это refsection=1.
        % Процитированные здесь работы:
        %  * подсчитываются, для автоматического составления фразы "Основные результаты ..."
        %  * попадают в авторскую библиографию, при usefootcite==0 и стиле `\insertbiblioauthor` или `\insertbiblioauthorgrouped`
        %  * нумеруются там в зависимости от порядка команд `\printbibliography` в этом разделе.
        %  * при использовании `\insertbiblioauthorgrouped`, порядок команд `\printbibliography` в нём должен быть тем же (см. biblio/biblatex.tex)
        %
        % Невидимый библиографический список для подсчёта количества публикаций:
        \ifxetexorluatex\selectlanguage{english}\fi
        \printbibliography[heading=nobibheading, section=1, env=countauthorvak,          keyword=biblioauthorvak]%
        \printbibliography[heading=nobibheading, section=1, env=countauthorwos,          keyword=biblioauthorwos]%
        \printbibliography[heading=nobibheading, section=1, env=countauthorscopus,       keyword=biblioauthorscopus]%
        \printbibliography[heading=nobibheading, section=1, env=countauthorconf,         keyword=biblioauthorconf]%
        \printbibliography[heading=nobibheading, section=1, env=countauthorother,        keyword=biblioauthorother]%
        \printbibliography[heading=nobibheading, section=1, env=countregistered,         keyword=biblioregistered]%
        \printbibliography[heading=nobibheading, section=1, env=countauthorpatent,       keyword=biblioauthorpatent]%
        \printbibliography[heading=nobibheading, section=1, env=countauthorprogram,      keyword=biblioauthorprogram]%
        \printbibliography[heading=nobibheading, section=1, env=countauthor,             keyword=biblioauthor]%
        \printbibliography[heading=nobibheading, section=1, env=countauthorvakscopuswos, filter=vakscopuswos]%
        \printbibliography[heading=nobibheading, section=1, env=countauthorscopuswos,    filter=scopuswos]%
        %
        \nocite{*}\ifxetexorluatex\selectlanguage{russian}\fi%
        %
        {\publications} Основные результаты по теме диссертации изложены в~\arabic{citeauthor}~печатных изданиях,
        \arabic{citeauthorvak} из которых изданы в журналах, рекомендованных ВАК\sloppy%
        \ifnum \value{citeauthorscopuswos}>0%
            , \arabic{citeauthorscopuswos} "--- в~периодических научных журналах, индексируемых Web of~Science и Scopus\sloppy%
        \fi%
        \ifnum \value{citeauthorconf}>0%
            , \arabic{citeauthorconf} "--- в~тезисах докладов.
        \else%
            .
        \fi%
        \ifnum \value{citeregistered}=1%
            \ifnum \value{citeauthorpatent}=1%
                Зарегистрирован \arabic{citeauthorpatent} патент.
            \fi%
            \ifnum \value{citeauthorprogram}=1%
                Зарегистрирована \arabic{citeauthorprogram} программа для ЭВМ.
            \fi%
        \fi%
        \ifnum \value{citeregistered}>1%
            Зарегистрированы\ %
            \ifnum \value{citeauthorpatent}>0%
            \formbytotal{citeauthorpatent}{патент}{}{а}{}\sloppy%
            \ifnum \value{citeauthorprogram}=0 . \else \ и~\fi%
            \fi%
            \ifnum \value{citeauthorprogram}>0%
            \formbytotal{citeauthorprogram}{программ}{а}{ы}{} для ЭВМ.
            \fi%
        \fi%
        % К публикациям, в которых излагаются основные научные результаты диссертации на соискание учёной
        % степени, в рецензируемых изданиях приравниваются патенты на изобретения, патенты (свидетельства) на
        % полезную модель, патенты на промышленный образец, патенты на селекционные достижения, свидетельства
        % на программу для электронных вычислительных машин, базу данных, топологию интегральных микросхем,
        % зарегистрированные в установленном порядке.(в ред. Постановления Правительства РФ от 21.04.2016 N 335)
    \end{refsection}%
    \begin{refsection}[bl-author, bl-registered]
        % Это refsection=2.
        % Процитированные здесь работы:
        %  * попадают в авторскую библиографию, при usefootcite==0 и стиле `\insertbiblioauthorimportant`.
        %  * ни на что не влияют в противном случае
        \nocite{vakbib2}%vak
        \nocite{patbib1}%patent
        \nocite{progbib1}%program
        \nocite{bib1}%other
        \nocite{confbib1}%conf
    \end{refsection}%
        %
        % Всё, что вне этих двух refsection, это refsection=0,
        %  * для диссертации - это нормальные ссылки, попадающие в обычную библиографию
        %  * для автореферата:
        %     * при usefootcite==0, ссылка корректно сработает только для источника из `external.bib`. Для своих работ --- напечатает "[0]" (и даже Warning не вылезет).
        %     * при usefootcite==1, ссылка сработает нормально. В авторской библиографии будут только процитированные в refsection=0 работы.
}

При использовании пакета \verb!biblatex! будут подсчитаны все работы, добавленные
в файл \verb!biblio/author.bib!. Для правильного подсчёта работ в~различных
системах цитирования требуется использовать поля:
\begin{itemize}
        \item \texttt{authorvak} если публикация индексирована ВАК,
        \item \texttt{authorscopus} если публикация индексирована Scopus,
        \item \texttt{authorwos} если публикация индексирована Web of Science,
        \item \texttt{authorconf} для докладов конференций,
        \item \texttt{authorpatent} для патентов,
        \item \texttt{authorprogram} для зарегистрированных программ для ЭВМ,
        \item \texttt{authorother} для других публикаций.
\end{itemize}
Для подсчёта используются счётчики:
\begin{itemize}
        \item \texttt{citeauthorvak} для работ, индексируемых ВАК,
        \item \texttt{citeauthorscopus} для работ, индексируемых Scopus,
        \item \texttt{citeauthorwos} для работ, индексируемых Web of Science,
        \item \texttt{citeauthorvakscopuswos} для работ, индексируемых одной из трёх баз,
        \item \texttt{citeauthorscopuswos} для работ, индексируемых Scopus или Web of~Science,
        \item \texttt{citeauthorconf} для докладов на конференциях,
        \item \texttt{citeauthorother} для остальных работ,
        \item \texttt{citeauthorpatent} для патентов,
        \item \texttt{citeauthorprogram} для зарегистрированных программ для ЭВМ,
        \item \texttt{citeauthor} для суммарного количества работ.
\end{itemize}
% Счётчик \texttt{citeexternal} используется для подсчёта процитированных публикаций;
% \texttt{citeregistered} "--- для подсчёта суммарного количества патентов и программ для ЭВМ.

Для добавления в список публикаций автора работ, которые не были процитированы в
автореферате, требуется их~перечислить с использованием команды \verb!\nocite! в
\verb!Synopsis/content.tex!.
