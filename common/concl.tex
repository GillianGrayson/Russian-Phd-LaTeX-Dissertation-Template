%% Согласно ГОСТ Р 7.0.11-2011:
%% 5.3.3 В заключении диссертации излагают итоги выполненного исследования, рекомендации, перспективы дальнейшей разработки темы.
%% 9.2.3 В заключении автореферата диссертации излагают итоги данного исследования, рекомендации и перспективы дальнейшей разработки темы.
\begin{enumerate}[beginpenalty=10000] % https://tex.stackexchange.com/a/476052/104425
	\item Разработан программный комплекс \cite{prog1}, осуществляющий численное моделирование открытых квантовых систем с большим числом состояний, включающий в себя возможность анализа отдельных квантовых траекторий, поиск асимптотических состояний системы путём численного интегрирования (при наличии модуляции в системе) или поиска собственного состояния системы.
	\item Обнаружено явление Андерсоновской локализации в открытых квантовых системах. Диссипация может быть использована для создания нетривиальных устойчивых состояний, в которых доминируют несколько локализованных андерсоновских мод пространственно-неоднородного гамильтониана \cite{Yusipov2017}.
	\item Был найден механизм управления асимптотическим состоянием в открытой квантовой системе с локализацией Андерсона. Данное состояние может быть локализовано в любом месте спектра гамильтониана, за счёт управляемой синтетической диссипации. Полученные таким образом состояния являются устойчивыми к дефазирующей диссипации \cite{Vershinina2017}.
	\item Были изучены различные типы распространения волновых пакетов открытой квантовой системе с локализацией Андерсона. В частности, баллистический режим вызванный суммарным взаимодействием беспорядка и диссипации \cite{Yusipov2018}. 
	\item Было обнаружено явление многочастичной локализации в открытых квантовых системах.
	Нетривиальная диссипация, действующая на соседних сайтах решётки приводит к корреляциям между далеко расположенными сайтами, и  даже достаточно слабая дефазирующая не может их разрушить. 
	В то же время механизмы многочастичной локализации, индуцированные гамильтонианом, пытаются ограничить корреляции длиной локализации. Были предложены новые количественные идентификаторы многочастичной локализации в открытых квантовых системах \cite{Vakulchyk2018}.
	\item Обнаружен квантовый аналог бифуркации Неймарка"--~Сакера (рождение тора из-за неустойчивости предельного цикла) в модели открытого периодически модулируемого димера \cite{Yusipov2019_1}.
	\item Разработан и реализован алгоритм нахождения старшего квантового показателя Ляпунова, основанный на методе квантовых траекторий. Данная разработка позволяет выявить сложную структуру регулярных и хаотических областей, дать количественную оценку диссипативному квантовому хаосу. Как и классический старший показатель Ляпунова, его квантовый аналог становится положительным в случае хаотической динамики системы \cite{Yusipov2019_2}.
	\item Предложен экспериментально осуществимый подход к обнаружению регулярных и хаотических режимов в определённом классе открытых квантовых систем на основе анализа статистики распределения времени ожидания фотона. При переходе в режим квантового хаоса режиме появляется степенная асимптотика в распределении времени ожидания фотона \cite{Yusipov2020}.
\end{enumerate}

Благодаря очень быстрому прогрессу в области прикладных квантовых вычислений сверхпроводящие чипы используются теперь не только для выполнения квантовых вычислений, но и для изучения многочастичной локализации и разнообразных фаз квантовых многочастичных систем, таким образом предоставляя новый, хорошо контролируемый и настраиваемый стенд для изучения сложных квантовых систем. 

Исследование диссипативного квантового хаоса и его механизмов позволит использовать диссипативные эффекты (вместо того, чтобы бороться и подавлять их) для создания принципиально новых режимов сверхпроводящих квантовых систем, решающих задачу устойчивой обработки квантовой информации на длительных временных масштабах.
