%% Согласно ГОСТ Р 7.0.11-2011:
%% 5.3.3 В заключении диссертации излагают итоги выполненного исследования, рекомендации, перспективы дальнейшей разработки темы.
%% 9.2.3 В заключении автореферата диссертации излагают итоги данного исследования, рекомендации и перспективы дальнейшей разработки темы.
\begin{enumerate}[beginpenalty=10000] % https://tex.stackexchange.com/a/476052/104425
	\item Разработан программный комплекс \cite{prog1}, осуществляющий численное моделирование открытых квантовых систем с большим числом состояний, включающий в себя возможность анализа отдельных квантовых траекторий, поиск асимптотических состояний системы путём численного интегрирования (при наличии модуляции в системе) или поиска собственного состояния системы.
	\item Обнаружена и численно исследована математическая модель открытой квантовой системы с признаками одночастичной локализации в асимптотических состояниях \cite{Yusipov2017}. 
	Разработан метод управления локализационными свойствами получаемого квантового аттрактора \cite{Vershinina2017} "--- асимптотическое состояние может быть локализовано в любом месте спектра гамильтониана, за счёт управляемой синтетической диссипации.
	В рассматриваемой модели изучены характеристики волновых пакетов отдельных квантовых траекторий в асимптотическом режиме \cite{Yusipov2018}.
	\item Численно исследована математическая модель открытой квантовой системы с признаками многочастичной локализации в асимптотических состояниях. Предложены количественные критерии данного явления \cite{Vakulchyk2018}.
	\item В ходе численного анализа модели открытого периодически модулируемого квантового димера обнаружен квантовый аналог бифуркации Неймарка"--~Сакера (рождение тора из-за неустойчивости предельного цикла) \cite{Yusipov2019_1}.
	\item Разработан и программно реализован \cite{prog1} алгоритм нахождения старшего квантового показателя Ляпунова, основанный на методе квантовых траекторий. Данная разработка позволяет выявить сложную структуру регулярных и хаотических областей, дать количественную оценку диссипативному квантовому хаосу. Как и классический старший показатель Ляпунова, его квантовый аналог становится положительным в случае хаотической динамики системы \cite{Yusipov2019_2}.
	\item Предложен экспериментально осуществимый подход к обнаружению регулярных и хаотических режимов в определённом классе математических моделей открытых квантовых систем на основе анализа статистики распределения времени ожидания фотона. При переходе в режим квантового хаоса режиме появляется степенная асимптотика в распределении времени ожидания фотона \cite{Yusipov2020}. Также численно исследованы математические модели со спин-фотонным взаимодействием, позволяющим регулировать динамику в системе \cite{Yusipov2021}.
\end{enumerate}
